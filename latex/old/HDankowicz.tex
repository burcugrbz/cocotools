
\documentclass{article}
\usepackage{amsfonts}

%%%%%%%%%%%%%%%%%%%%%%%%%%%%%%%%%%%%%%%%%%%%%%%%%%%%%%%%%%%%%%%%%%%%%%%%%%%%%%%%%%%%%%%%%%%%%%%%%%%
%TCIDATA{OutputFilter=LATEX.DLL}
%TCIDATA{Created=Friday, February 08, 2008 13:22:05}
%TCIDATA{LastRevised=Saturday, August 16, 2008 16:04:28}
%TCIDATA{<META NAME="GraphicsSave" CONTENT="32">}
%TCIDATA{<META NAME="DocumentShell" CONTENT="General\SW\Blank - Standard LaTeX Article">}
%TCIDATA{CSTFile=LaTeX article (bright).cst}

\newtheorem{theorem}{Theorem}
\newtheorem{acknowledgement}[theorem]{Acknowledgement}
\newtheorem{algorithm}[theorem]{Algorithm}
\newtheorem{axiom}[theorem]{Axiom}
\newtheorem{case}[theorem]{Case}
\newtheorem{claim}[theorem]{Claim}
\newtheorem{conclusion}[theorem]{Conclusion}
\newtheorem{condition}[theorem]{Condition}
\newtheorem{conjecture}[theorem]{Conjecture}
\newtheorem{corollary}[theorem]{Corollary}
\newtheorem{criterion}[theorem]{Criterion}
\newtheorem{definition}[theorem]{Definition}
\newtheorem{example}[theorem]{Example}
\newtheorem{exercise}[theorem]{Exercise}
\newtheorem{lemma}[theorem]{Lemma}
\newtheorem{notation}[theorem]{Notation}
\newtheorem{problem}[theorem]{Problem}
\newtheorem{proposition}[theorem]{Proposition}
\newtheorem{remark}[theorem]{Remark}
\newtheorem{solution}[theorem]{Solution}
\newtheorem{summary}[theorem]{Summary}
\newenvironment{proof}[1][Proof]{\textbf{#1.} }{\ \rule{0.5em}{0.5em}}
% Macros for Scientific Word 3.0 documents saved with the LaTeX filter.
%Copyright (C) 1994-97 TCI Software Research, Inc.
\typeout{TCILATEX Macros for Scientific Word 3.0 <19 May 1997>.}
\typeout{NOTICE:  This macro file is NOT proprietary and may be 
freely copied and distributed.}
%
\makeatletter
%
%%%%%%%%%%%%%%%%%%%%%%
% macros for time
\newcount\@hour\newcount\@minute\chardef\@x10\chardef\@xv60
\def\tcitime{
\def\@time{%
  \@minute\time\@hour\@minute\divide\@hour\@xv
  \ifnum\@hour<\@x 0\fi\the\@hour:%
  \multiply\@hour\@xv\advance\@minute-\@hour
  \ifnum\@minute<\@x 0\fi\the\@minute
  }}%

%%%%%%%%%%%%%%%%%%%%%%
% macro for hyperref
\@ifundefined{hyperref}{\def\hyperref#1#2#3#4{#2\ref{#4}#3}}{}

% macro for external program call
\@ifundefined{qExtProgCall}{\def\qExtProgCall#1#2#3#4#5#6{\relax}}{}
%%%%%%%%%%%%%%%%%%%%%%
%
% macros for graphics
%
\def\FILENAME#1{#1}%
%
\def\QCTOpt[#1]#2{%
  \def\QCTOptB{#1}
  \def\QCTOptA{#2}
}
\def\QCTNOpt#1{%
  \def\QCTOptA{#1}
  \let\QCTOptB\empty
}
\def\Qct{%
  \@ifnextchar[{%
    \QCTOpt}{\QCTNOpt}
}
\def\QCBOpt[#1]#2{%
  \def\QCBOptB{#1}
  \def\QCBOptA{#2}
}
\def\QCBNOpt#1{%
  \def\QCBOptA{#1}
  \let\QCBOptB\empty
}
\def\Qcb{%
  \@ifnextchar[{%
    \QCBOpt}{\QCBNOpt}
}
\def\PrepCapArgs{%
  \ifx\QCBOptA\empty
    \ifx\QCTOptA\empty
      {}%
    \else
      \ifx\QCTOptB\empty
        {\QCTOptA}%
      \else
        [\QCTOptB]{\QCTOptA}%
      \fi
    \fi
  \else
    \ifx\QCBOptA\empty
      {}%
    \else
      \ifx\QCBOptB\empty
        {\QCBOptA}%
      \else
        [\QCBOptB]{\QCBOptA}%
      \fi
    \fi
  \fi
}
\newcount\GRAPHICSTYPE
%\GRAPHICSTYPE 0 is for TurboTeX
%\GRAPHICSTYPE 1 is for DVIWindo (PostScript)
%%%(removed)%\GRAPHICSTYPE 2 is for psfig (PostScript)
\GRAPHICSTYPE=\z@
\def\GRAPHICSPS#1{%
 \ifcase\GRAPHICSTYPE%\GRAPHICSTYPE=0
   \special{ps: #1}%
 \or%\GRAPHICSTYPE=1
   \special{language "PS", include "#1"}%
%%%\or%\GRAPHICSTYPE=2
%%%  #1%
 \fi
}%
%
\def\GRAPHICSHP#1{\special{include #1}}%
%
% \graffile{ body }                                  %#1
%          { contentswidth (scalar)  }               %#2
%          { contentsheight (scalar) }               %#3
%          { vertical shift when in-line (scalar) }  %#4
\def\graffile#1#2#3#4{%
%%% \ifnum\GRAPHICSTYPE=\tw@
%%%  %Following if using psfig
%%%  \@ifundefined{psfig}{\input psfig.tex}{}%
%%%  \psfig{file=#1, height=#3, width=#2}%
%%% \else
  %Following for all others
  % JCS - added BOXTHEFRAME, see below
    \bgroup
    \leavevmode
    \@ifundefined{bbl@deactivate}{\def~{\string~}}{\activesoff}
    \raise -#4 \BOXTHEFRAME{%
        \hbox to #2{\raise #3\hbox to #2{\null #1\hfil}}}%
    \egroup
}%
%
% A box for drafts
\def\draftbox#1#2#3#4{%
 \leavevmode\raise -#4 \hbox{%
  \frame{\rlap{\protect\tiny #1}\hbox to #2%
   {\vrule height#3 width\z@ depth\z@\hfil}%
  }%
 }%
}%
%
\newcount\draft
\draft=\z@
\let\nographics=\draft
\newif\ifwasdraft
\wasdraftfalse

%  \GRAPHIC{ body }                                  %#1
%          { draft name }                            %#2
%          { contentswidth (scalar)  }               %#3
%          { contentsheight (scalar) }               %#4
%          { vertical shift when in-line (scalar) }  %#5
\def\GRAPHIC#1#2#3#4#5{%
 \ifnum\draft=\@ne\draftbox{#2}{#3}{#4}{#5}%
  \else\graffile{#1}{#3}{#4}{#5}%
  \fi
 }%
%
\def\addtoLaTeXparams#1{%
    \edef\LaTeXparams{\LaTeXparams #1}}%
%
% JCS -  added a switch BoxFrame that can 
% be set by including X in the frame params.
% If set a box is drawn around the frame.

\newif\ifBoxFrame \BoxFramefalse
\newif\ifOverFrame \OverFramefalse
\newif\ifUnderFrame \UnderFramefalse

\def\BOXTHEFRAME#1{%
   \hbox{%
      \ifBoxFrame
         \frame{#1}%
      \else
         {#1}%
      \fi
   }%
}


\def\doFRAMEparams#1{\BoxFramefalse\OverFramefalse\UnderFramefalse\readFRAMEparams#1\end}%
\def\readFRAMEparams#1{%
 \ifx#1\end%
  \let\next=\relax
  \else
  \ifx#1i\dispkind=\z@\fi
  \ifx#1d\dispkind=\@ne\fi
  \ifx#1f\dispkind=\tw@\fi
  \ifx#1t\addtoLaTeXparams{t}\fi
  \ifx#1b\addtoLaTeXparams{b}\fi
  \ifx#1p\addtoLaTeXparams{p}\fi
  \ifx#1h\addtoLaTeXparams{h}\fi
  \ifx#1X\BoxFrametrue\fi
  \ifx#1O\OverFrametrue\fi
  \ifx#1U\UnderFrametrue\fi
  \ifx#1w
    \ifnum\draft=1\wasdrafttrue\else\wasdraftfalse\fi
    \draft=\@ne
  \fi
  \let\next=\readFRAMEparams
  \fi
 \next
 }%
%
%Macro for In-line graphics object
%   \IFRAME{ contentswidth (scalar)  }               %#1
%          { contentsheight (scalar) }               %#2
%          { vertical shift when in-line (scalar) }  %#3
%          { draft name }                            %#4
%          { body }                                  %#5
%          { caption}                                %#6


\def\IFRAME#1#2#3#4#5#6{%
      \bgroup
      \let\QCTOptA\empty
      \let\QCTOptB\empty
      \let\QCBOptA\empty
      \let\QCBOptB\empty
      #6%
      \parindent=0pt%
      \leftskip=0pt
      \rightskip=0pt
      \setbox0 = \hbox{\QCBOptA}%
      \@tempdima = #1\relax
      \ifOverFrame
          % Do this later
          \typeout{This is not implemented yet}%
          \show\HELP
      \else
         \ifdim\wd0>\@tempdima
            \advance\@tempdima by \@tempdima
            \ifdim\wd0 >\@tempdima
               \textwidth=\@tempdima
               \setbox1 =\vbox{%
                  \noindent\hbox to \@tempdima{\hfill\GRAPHIC{#5}{#4}{#1}{#2}{#3}\hfill}\\%
                  \noindent\hbox to \@tempdima{\parbox[b]{\@tempdima}{\QCBOptA}}%
               }%
               \wd1=\@tempdima
            \else
               \textwidth=\wd0
               \setbox1 =\vbox{%
                 \noindent\hbox to \wd0{\hfill\GRAPHIC{#5}{#4}{#1}{#2}{#3}\hfill}\\%
                 \noindent\hbox{\QCBOptA}%
               }%
               \wd1=\wd0
            \fi
         \else
            %\show\BBB
            \ifdim\wd0>0pt
              \hsize=\@tempdima
              \setbox1 =\vbox{%
                \unskip\GRAPHIC{#5}{#4}{#1}{#2}{0pt}%
                \break
                \unskip\hbox to \@tempdima{\hfill \QCBOptA\hfill}%
              }%
              \wd1=\@tempdima
           \else
              \hsize=\@tempdima
              \setbox1 =\vbox{%
                \unskip\GRAPHIC{#5}{#4}{#1}{#2}{0pt}%
              }%
              \wd1=\@tempdima
           \fi
         \fi
         \@tempdimb=\ht1
         \advance\@tempdimb by \dp1
         \advance\@tempdimb by -#2%
         \advance\@tempdimb by #3%
         \leavevmode
         \raise -\@tempdimb \hbox{\box1}%
      \fi
      \egroup%
}%
%
%Macro for Display graphics object
%   \DFRAME{ contentswidth (scalar)  }               %#1
%          { contentsheight (scalar) }               %#2
%          { draft label }                           %#3
%          { name }                                  %#4
%          { caption}                                %#5
\def\DFRAME#1#2#3#4#5{%
 \begin{center}
     \let\QCTOptA\empty
     \let\QCTOptB\empty
     \let\QCBOptA\empty
     \let\QCBOptB\empty
     \ifOverFrame 
        #5\QCTOptA\par
     \fi
     \GRAPHIC{#4}{#3}{#1}{#2}{\z@}
     \ifUnderFrame 
        \nobreak\par\nobreak#5\QCBOptA
     \fi
 \end{center}%
 }%
%
%Macro for Floating graphic object
%   \FFRAME{ framedata f|i tbph x F|T }              %#1
%          { contentswidth (scalar)  }               %#2
%          { contentsheight (scalar) }               %#3
%          { caption }                               %#4
%          { label }                                 %#5
%          { draft name }                            %#6
%          { body }                                  %#7
\def\FFRAME#1#2#3#4#5#6#7{%
 \begin{figure}[#1]%
  \let\QCTOptA\empty
  \let\QCTOptB\empty
  \let\QCBOptA\empty
  \let\QCBOptB\empty
  \ifOverFrame
    #4
    \ifx\QCTOptA\empty
    \else
      \ifx\QCTOptB\empty
        \caption{\QCTOptA}%
      \else
        \caption[\QCTOptB]{\QCTOptA}%
      \fi
    \fi
    \ifUnderFrame\else
      \label{#5}%
    \fi
  \else
    \UnderFrametrue%
  \fi
  \begin{center}\GRAPHIC{#7}{#6}{#2}{#3}{\z@}\end{center}%
  \ifUnderFrame
    #4
    \ifx\QCBOptA\empty
      \caption{}%
    \else
      \ifx\QCBOptB\empty
        \caption{\QCBOptA}%
      \else
        \caption[\QCBOptB]{\QCBOptA}%
      \fi
    \fi
    \label{#5}%
  \fi
  \end{figure}%
 }%
%
%
%    \FRAME{ framedata f|i tbph x F|T }              %#1
%          { contentswidth (scalar)  }               %#2
%          { contentsheight (scalar) }               %#3
%          { vertical shift when in-line (scalar) }  %#4
%          { caption }                               %#5
%          { label }                                 %#6
%          { name }                                  %#7
%          { body }                                  %#8
%
%    framedata is a string which can contain the following
%    characters: idftbphxFT
%    Their meaning is as follows:
%             i, d or f : in-line, display, or floating
%             t,b,p,h   : LaTeX floating placement options
%             x         : fit contents box to contents
%             F or T    : Figure or Table. 
%                         Later this can expand
%                         to a more general float class.
%
%
\newcount\dispkind%

\def\makeactives{
  \catcode`\"=\active
  \catcode`\;=\active
  \catcode`\:=\active
  \catcode`\'=\active
  \catcode`\~=\active
}
\bgroup
   \makeactives
   \gdef\activesoff{%
      \def"{\string"}
      \def;{\string;}
      \def:{\string:}
      \def'{\string'}
      \def~{\string~}
      %\bbl@deactivate{"}%
      %\bbl@deactivate{;}%
      %\bbl@deactivate{:}%
      %\bbl@deactivate{'}%
    }
\egroup

\def\FRAME#1#2#3#4#5#6#7#8{%
 \bgroup
 \ifnum\draft=\@ne
   \wasdrafttrue
 \else
   \wasdraftfalse%
 \fi
 \def\LaTeXparams{}%
 \dispkind=\z@
 \def\LaTeXparams{}%
 \doFRAMEparams{#1}%
 \ifnum\dispkind=\z@\IFRAME{#2}{#3}{#4}{#7}{#8}{#5}\else
  \ifnum\dispkind=\@ne\DFRAME{#2}{#3}{#7}{#8}{#5}\else
   \ifnum\dispkind=\tw@
    \edef\@tempa{\noexpand\FFRAME{\LaTeXparams}}%
    \@tempa{#2}{#3}{#5}{#6}{#7}{#8}%
    \fi
   \fi
  \fi
  \ifwasdraft\draft=1\else\draft=0\fi{}%
  \egroup
 }%
%
% This macro added to let SW gobble a parameter that
% should not be passed on and expanded. 

\def\TEXUX#1{"texux"}

%
% Macros for text attributes:
%
\def\BF#1{{\bf {#1}}}%
\def\NEG#1{\leavevmode\hbox{\rlap{\thinspace/}{$#1$}}}%
%
%%%%%%%%%%%%%%%%%%%%%%%%%%%%%%%%%%%%%%%%%%%%%%%%%%%%%%%%%%%%%%%%%%%%%%%%
%
%
% macros for user - defined functions
\def\limfunc#1{\mathop{\rm #1}}%
\def\func#1{\mathop{\rm #1}\nolimits}%
% macro for unit names
\def\unit#1{\mathop{\rm #1}\nolimits}%

%
% miscellaneous 
\long\def\QQQ#1#2{%
     \long\expandafter\def\csname#1\endcsname{#2}}%
\@ifundefined{QTP}{\def\QTP#1{}}{}
\@ifundefined{QEXCLUDE}{\def\QEXCLUDE#1{}}{}
\@ifundefined{Qlb}{\def\Qlb#1{#1}}{}
\@ifundefined{Qlt}{\def\Qlt#1{#1}}{}
\def\QWE{}%
\long\def\QQA#1#2{}%
\def\QTR#1#2{{\csname#1\endcsname #2}}%(gp) Is this the best?
\long\def\TeXButton#1#2{#2}%
\long\def\QSubDoc#1#2{#2}%
\def\EXPAND#1[#2]#3{}%
\def\NOEXPAND#1[#2]#3{}%
\def\PROTECTED{}%
\def\LaTeXparent#1{}%
\def\ChildStyles#1{}%
\def\ChildDefaults#1{}%
\def\QTagDef#1#2#3{}%

% Constructs added with Scientific Notebook
\@ifundefined{correctchoice}{\def\correctchoice{\relax}}{}
\@ifundefined{HTML}{\def\HTML#1{\relax}}{}
\@ifundefined{TCIIcon}{\def\TCIIcon#1#2#3#4{\relax}}{}
\if@compatibility
  \typeout{Not defining UNICODE or CustomNote commands for LaTeX 2.09.}
\else
  \providecommand{\UNICODE}[2][]{}
  \providecommand{\CustomNote}[3][]{\marginpar{#3}}
\fi

%
% Macros for style editor docs
\@ifundefined{StyleEditBeginDoc}{\def\StyleEditBeginDoc{\relax}}{}
%
% Macros for footnotes
\def\QQfnmark#1{\footnotemark}
\def\QQfntext#1#2{\addtocounter{footnote}{#1}\footnotetext{#2}}
%
% Macros for indexing.
%
\@ifundefined{TCIMAKEINDEX}{}{\makeindex}%
%
% Attempts to avoid problems with other styles
\@ifundefined{abstract}{%
 \def\abstract{%
  \if@twocolumn
   \section*{Abstract (Not appropriate in this style!)}%
   \else \small 
   \begin{center}{\bf Abstract\vspace{-.5em}\vspace{\z@}}\end{center}%
   \quotation 
   \fi
  }%
 }{%
 }%
\@ifundefined{endabstract}{\def\endabstract
  {\if@twocolumn\else\endquotation\fi}}{}%
\@ifundefined{maketitle}{\def\maketitle#1{}}{}%
\@ifundefined{affiliation}{\def\affiliation#1{}}{}%
\@ifundefined{proof}{\def\proof{\noindent{\bfseries Proof. }}}{}%
\@ifundefined{endproof}{\def\endproof{\mbox{\ \rule{.1in}{.1in}}}}{}%
\@ifundefined{newfield}{\def\newfield#1#2{}}{}%
\@ifundefined{chapter}{\def\chapter#1{\par(Chapter head:)#1\par }%
 \newcount\c@chapter}{}%
\@ifundefined{part}{\def\part#1{\par(Part head:)#1\par }}{}%
\@ifundefined{section}{\def\section#1{\par(Section head:)#1\par }}{}%
\@ifundefined{subsection}{\def\subsection#1%
 {\par(Subsection head:)#1\par }}{}%
\@ifundefined{subsubsection}{\def\subsubsection#1%
 {\par(Subsubsection head:)#1\par }}{}%
\@ifundefined{paragraph}{\def\paragraph#1%
 {\par(Subsubsubsection head:)#1\par }}{}%
\@ifundefined{subparagraph}{\def\subparagraph#1%
 {\par(Subsubsubsubsection head:)#1\par }}{}%
%%%%%%%%%%%%%%%%%%%%%%%%%%%%%%%%%%%%%%%%%%%%%%%%%%%%%%%%%%%%%%%%%%%%%%%%
% These symbols are not recognized by LaTeX
\@ifundefined{therefore}{\def\therefore{}}{}%
\@ifundefined{backepsilon}{\def\backepsilon{}}{}%
\@ifundefined{yen}{\def\yen{\hbox{\rm\rlap=Y}}}{}%
\@ifundefined{registered}{%
   \def\registered{\relax\ifmmode{}\r@gistered
                    \else$\m@th\r@gistered$\fi}%
 \def\r@gistered{^{\ooalign
  {\hfil\raise.07ex\hbox{$\scriptstyle\rm\text{R}$}\hfil\crcr
  \mathhexbox20D}}}}{}%
\@ifundefined{Eth}{\def\Eth{}}{}%
\@ifundefined{eth}{\def\eth{}}{}%
\@ifundefined{Thorn}{\def\Thorn{}}{}%
\@ifundefined{thorn}{\def\thorn{}}{}%
% A macro to allow any symbol that requires math to appear in text
\def\TEXTsymbol#1{\mbox{$#1$}}%
\@ifundefined{degree}{\def\degree{{}^{\circ}}}{}%
%
% macros for T3TeX files
\newdimen\theight
\def\Column{%
 \vadjust{\setbox\z@=\hbox{\scriptsize\quad\quad tcol}%
  \theight=\ht\z@\advance\theight by \dp\z@\advance\theight by \lineskip
  \kern -\theight \vbox to \theight{%
   \rightline{\rlap{\box\z@}}%
   \vss
   }%
  }%
 }%
%
\def\qed{%
 \ifhmode\unskip\nobreak\fi\ifmmode\ifinner\else\hskip5\p@\fi\fi
 \hbox{\hskip5\p@\vrule width4\p@ height6\p@ depth1.5\p@\hskip\p@}%
 }%
%
\def\cents{\hbox{\rm\rlap/c}}%
\def\miss{\hbox{\vrule height2\p@ width 2\p@ depth\z@}}%
%
\def\vvert{\Vert}%           %always translated to \left| or \right|
%
\def\tcol#1{{\baselineskip=6\p@ \vcenter{#1}} \Column}  %
%
\def\dB{\hbox{{}}}%                 %dummy entry in column 
\def\mB#1{\hbox{$#1$}}%             %column entry
\def\nB#1{\hbox{#1}}%               %column entry (not math)
%
\@ifundefined{note}{\def\note{$^{\dag}}}{}%
%

\def\newfmtname{LaTeX2e}
% No longer load latexsym.  This is now handled by SWP, which uses amsfonts if necessary

\ifx\fmtname\newfmtname
  \DeclareOldFontCommand{\rm}{\normalfont\rmfamily}{\mathrm}
  \DeclareOldFontCommand{\sf}{\normalfont\sffamily}{\mathsf}
  \DeclareOldFontCommand{\tt}{\normalfont\ttfamily}{\mathtt}
  \DeclareOldFontCommand{\bf}{\normalfont\bfseries}{\mathbf}
  \DeclareOldFontCommand{\it}{\normalfont\itshape}{\mathit}
  \DeclareOldFontCommand{\sl}{\normalfont\slshape}{\@nomath\sl}
  \DeclareOldFontCommand{\sc}{\normalfont\scshape}{\@nomath\sc}
\fi

%
% Greek bold macros
% Redefine all of the math symbols 
% which might be bolded	 - there are 
% probably others to add to this list

\def\alpha{{\Greekmath 010B}}%
\def\beta{{\Greekmath 010C}}%
\def\gamma{{\Greekmath 010D}}%
\def\delta{{\Greekmath 010E}}%
\def\epsilon{{\Greekmath 010F}}%
\def\zeta{{\Greekmath 0110}}%
\def\eta{{\Greekmath 0111}}%
\def\theta{{\Greekmath 0112}}%
\def\iota{{\Greekmath 0113}}%
\def\kappa{{\Greekmath 0114}}%
\def\lambda{{\Greekmath 0115}}%
\def\mu{{\Greekmath 0116}}%
\def\nu{{\Greekmath 0117}}%
\def\xi{{\Greekmath 0118}}%
\def\pi{{\Greekmath 0119}}%
\def\rho{{\Greekmath 011A}}%
\def\sigma{{\Greekmath 011B}}%
\def\tau{{\Greekmath 011C}}%
\def\upsilon{{\Greekmath 011D}}%
\def\phi{{\Greekmath 011E}}%
\def\chi{{\Greekmath 011F}}%
\def\psi{{\Greekmath 0120}}%
\def\omega{{\Greekmath 0121}}%
\def\varepsilon{{\Greekmath 0122}}%
\def\vartheta{{\Greekmath 0123}}%
\def\varpi{{\Greekmath 0124}}%
\def\varrho{{\Greekmath 0125}}%
\def\varsigma{{\Greekmath 0126}}%
\def\varphi{{\Greekmath 0127}}%

\def\nabla{{\Greekmath 0272}}
\def\FindBoldGroup{%
   {\setbox0=\hbox{$\mathbf{x\global\edef\theboldgroup{\the\mathgroup}}$}}%
}

\def\Greekmath#1#2#3#4{%
    \if@compatibility
        \ifnum\mathgroup=\symbold
           \mathchoice{\mbox{\boldmath$\displaystyle\mathchar"#1#2#3#4$}}%
                      {\mbox{\boldmath$\textstyle\mathchar"#1#2#3#4$}}%
                      {\mbox{\boldmath$\scriptstyle\mathchar"#1#2#3#4$}}%
                      {\mbox{\boldmath$\scriptscriptstyle\mathchar"#1#2#3#4$}}%
        \else
           \mathchar"#1#2#3#4% 
        \fi 
    \else 
        \FindBoldGroup
        \ifnum\mathgroup=\theboldgroup % For 2e
           \mathchoice{\mbox{\boldmath$\displaystyle\mathchar"#1#2#3#4$}}%
                      {\mbox{\boldmath$\textstyle\mathchar"#1#2#3#4$}}%
                      {\mbox{\boldmath$\scriptstyle\mathchar"#1#2#3#4$}}%
                      {\mbox{\boldmath$\scriptscriptstyle\mathchar"#1#2#3#4$}}%
        \else
           \mathchar"#1#2#3#4% 
        \fi     	    
	  \fi}

\newif\ifGreekBold  \GreekBoldfalse
\let\SAVEPBF=\pbf
\def\pbf{\GreekBoldtrue\SAVEPBF}%
%

\@ifundefined{theorem}{\newtheorem{theorem}{Theorem}}{}
\@ifundefined{lemma}{\newtheorem{lemma}[theorem]{Lemma}}{}
\@ifundefined{corollary}{\newtheorem{corollary}[theorem]{Corollary}}{}
\@ifundefined{conjecture}{\newtheorem{conjecture}[theorem]{Conjecture}}{}
\@ifundefined{proposition}{\newtheorem{proposition}[theorem]{Proposition}}{}
\@ifundefined{axiom}{\newtheorem{axiom}{Axiom}}{}
\@ifundefined{remark}{\newtheorem{remark}{Remark}}{}
\@ifundefined{example}{\newtheorem{example}{Example}}{}
\@ifundefined{exercise}{\newtheorem{exercise}{Exercise}}{}
\@ifundefined{definition}{\newtheorem{definition}{Definition}}{}


\@ifundefined{mathletters}{%
  %\def\theequation{\arabic{equation}}
  \newcounter{equationnumber}  
  \def\mathletters{%
     \addtocounter{equation}{1}
     \edef\@currentlabel{\theequation}%
     \setcounter{equationnumber}{\c@equation}
     \setcounter{equation}{0}%
     \edef\theequation{\@currentlabel\noexpand\alph{equation}}%
  }
  \def\endmathletters{%
     \setcounter{equation}{\value{equationnumber}}%
  }
}{}

%Logos
\@ifundefined{BibTeX}{%
    \def\BibTeX{{\rm B\kern-.05em{\sc i\kern-.025em b}\kern-.08em
                 T\kern-.1667em\lower.7ex\hbox{E}\kern-.125emX}}}{}%
\@ifundefined{AmS}%
    {\def\AmS{{\protect\usefont{OMS}{cmsy}{m}{n}%
                A\kern-.1667em\lower.5ex\hbox{M}\kern-.125emS}}}{}%
\@ifundefined{AmSTeX}{\def\AmSTeX{\protect\AmS-\protect\TeX\@}}{}%
%

% This macro is a fix to eqnarray
\def\@@eqncr{\let\@tempa\relax
    \ifcase\@eqcnt \def\@tempa{& & &}\or \def\@tempa{& &}%
      \else \def\@tempa{&}\fi
     \@tempa
     \if@eqnsw
        \iftag@
           \@taggnum
        \else
           \@eqnnum\stepcounter{equation}%
        \fi
     \fi
     \global\tag@false
     \global\@eqnswtrue
     \global\@eqcnt\z@\cr}


\def\TCItag{\@ifnextchar*{\@TCItagstar}{\@TCItag}}
\def\@TCItag#1{%
    \global\tag@true
    \global\def\@taggnum{(#1)}}
\def\@TCItagstar*#1{%
    \global\tag@true
    \global\def\@taggnum{#1}}
%
%%%%%%%%%%%%%%%%%%%%%%%%%%%%%%%%%%%%%%%%%%%%%%%%%%%%%%%%%%%%%%%%%%%%%
%
\def\tfrac#1#2{{\textstyle {#1 \over #2}}}%
\def\dfrac#1#2{{\displaystyle {#1 \over #2}}}%
\def\binom#1#2{{#1 \choose #2}}%
\def\tbinom#1#2{{\textstyle {#1 \choose #2}}}%
\def\dbinom#1#2{{\displaystyle {#1 \choose #2}}}%
\def\QATOP#1#2{{#1 \atop #2}}%
\def\QTATOP#1#2{{\textstyle {#1 \atop #2}}}%
\def\QDATOP#1#2{{\displaystyle {#1 \atop #2}}}%
\def\QABOVE#1#2#3{{#2 \above#1 #3}}%
\def\QTABOVE#1#2#3{{\textstyle {#2 \above#1 #3}}}%
\def\QDABOVE#1#2#3{{\displaystyle {#2 \above#1 #3}}}%
\def\QOVERD#1#2#3#4{{#3 \overwithdelims#1#2 #4}}%
\def\QTOVERD#1#2#3#4{{\textstyle {#3 \overwithdelims#1#2 #4}}}%
\def\QDOVERD#1#2#3#4{{\displaystyle {#3 \overwithdelims#1#2 #4}}}%
\def\QATOPD#1#2#3#4{{#3 \atopwithdelims#1#2 #4}}%
\def\QTATOPD#1#2#3#4{{\textstyle {#3 \atopwithdelims#1#2 #4}}}%
\def\QDATOPD#1#2#3#4{{\displaystyle {#3 \atopwithdelims#1#2 #4}}}%
\def\QABOVED#1#2#3#4#5{{#4 \abovewithdelims#1#2#3 #5}}%
\def\QTABOVED#1#2#3#4#5{{\textstyle 
   {#4 \abovewithdelims#1#2#3 #5}}}%
\def\QDABOVED#1#2#3#4#5{{\displaystyle 
   {#4 \abovewithdelims#1#2#3 #5}}}%
%
% Macros for text size operators:
%
\def\tint{\mathop{\textstyle \int}}%
\def\tiint{\mathop{\textstyle \iint }}%
\def\tiiint{\mathop{\textstyle \iiint }}%
\def\tiiiint{\mathop{\textstyle \iiiint }}%
\def\tidotsint{\mathop{\textstyle \idotsint }}%
\def\toint{\mathop{\textstyle \oint}}%
\def\tsum{\mathop{\textstyle \sum }}%
\def\tprod{\mathop{\textstyle \prod }}%
\def\tbigcap{\mathop{\textstyle \bigcap }}%
\def\tbigwedge{\mathop{\textstyle \bigwedge }}%
\def\tbigoplus{\mathop{\textstyle \bigoplus }}%
\def\tbigodot{\mathop{\textstyle \bigodot }}%
\def\tbigsqcup{\mathop{\textstyle \bigsqcup }}%
\def\tcoprod{\mathop{\textstyle \coprod }}%
\def\tbigcup{\mathop{\textstyle \bigcup }}%
\def\tbigvee{\mathop{\textstyle \bigvee }}%
\def\tbigotimes{\mathop{\textstyle \bigotimes }}%
\def\tbiguplus{\mathop{\textstyle \biguplus }}%
%
%
%Macros for display size operators:
%
\def\dint{\mathop{\displaystyle \int}}%
\def\diint{\mathop{\displaystyle \iint }}%
\def\diiint{\mathop{\displaystyle \iiint }}%
\def\diiiint{\mathop{\displaystyle \iiiint }}%
\def\didotsint{\mathop{\displaystyle \idotsint }}%
\def\doint{\mathop{\displaystyle \oint}}%
\def\dsum{\mathop{\displaystyle \sum }}%
\def\dprod{\mathop{\displaystyle \prod }}%
\def\dbigcap{\mathop{\displaystyle \bigcap }}%
\def\dbigwedge{\mathop{\displaystyle \bigwedge }}%
\def\dbigoplus{\mathop{\displaystyle \bigoplus }}%
\def\dbigodot{\mathop{\displaystyle \bigodot }}%
\def\dbigsqcup{\mathop{\displaystyle \bigsqcup }}%
\def\dcoprod{\mathop{\displaystyle \coprod }}%
\def\dbigcup{\mathop{\displaystyle \bigcup }}%
\def\dbigvee{\mathop{\displaystyle \bigvee }}%
\def\dbigotimes{\mathop{\displaystyle \bigotimes }}%
\def\dbiguplus{\mathop{\displaystyle \biguplus }}%

%%%%%%%%%%%%%%%%%%%%%%%%%%%%%%%%%%%%%%%%%%%%%%%%%%%%%%%%%%%%%%%%%%%%%%%
% NOTE: The rest of this file is read only if amstex has not been
% loaded.  This section is used to define amstex constructs in the
% event they have not been defined.
%
%
\ifx\ds@amstex\relax
   \message{amstex already loaded}\makeatother\endinput% 2.09 compatability
\else
   \@ifpackageloaded{amsmath}%
      {\message{amsmath already loaded}\makeatother\endinput}
      {}
   \@ifpackageloaded{amstex}%
      {\message{amstex already loaded}\makeatother\endinput}
      {}
   \@ifpackageloaded{amsgen}%
      {\message{amsgen already loaded}\makeatother\endinput}
      {}
\fi
%%%%%%%%%%%%%%%%%%%%%%%%%%%%%%%%%%%%%%%%%%%%%%%%%%%%%%%%%%%%%%%%%%%%%%%%
%%
%
%
%  Macros to define some AMS LaTeX constructs when 
%  AMS LaTeX has not been loaded
% 
% These macros are copied from the AMS-TeX package for doing
% multiple integrals.
%
\let\DOTSI\relax
\def\RIfM@{\relax\ifmmode}%
\def\FN@{\futurelet\next}%
\newcount\intno@
\def\iint{\DOTSI\intno@\tw@\FN@\ints@}%
\def\iiint{\DOTSI\intno@\thr@@\FN@\ints@}%
\def\iiiint{\DOTSI\intno@4 \FN@\ints@}%
\def\idotsint{\DOTSI\intno@\z@\FN@\ints@}%
\def\ints@{\findlimits@\ints@@}%
\newif\iflimtoken@
\newif\iflimits@
\def\findlimits@{\limtoken@true\ifx\next\limits\limits@true
 \else\ifx\next\nolimits\limits@false\else
 \limtoken@false\ifx\ilimits@\nolimits\limits@false\else
 \ifinner\limits@false\else\limits@true\fi\fi\fi\fi}%
\def\multint@{\int\ifnum\intno@=\z@\intdots@                          %1
 \else\intkern@\fi                                                    %2
 \ifnum\intno@>\tw@\int\intkern@\fi                                   %3
 \ifnum\intno@>\thr@@\int\intkern@\fi                                 %4
 \int}%                                                               %5
\def\multintlimits@{\intop\ifnum\intno@=\z@\intdots@\else\intkern@\fi
 \ifnum\intno@>\tw@\intop\intkern@\fi
 \ifnum\intno@>\thr@@\intop\intkern@\fi\intop}%
\def\intic@{%
    \mathchoice{\hskip.5em}{\hskip.4em}{\hskip.4em}{\hskip.4em}}%
\def\negintic@{\mathchoice
 {\hskip-.5em}{\hskip-.4em}{\hskip-.4em}{\hskip-.4em}}%
\def\ints@@{\iflimtoken@                                              %1
 \def\ints@@@{\iflimits@\negintic@
   \mathop{\intic@\multintlimits@}\limits                             %2
  \else\multint@\nolimits\fi                                          %3
  \eat@}%                                                             %4
 \else                                                                %5
 \def\ints@@@{\iflimits@\negintic@
  \mathop{\intic@\multintlimits@}\limits\else
  \multint@\nolimits\fi}\fi\ints@@@}%
\def\intkern@{\mathchoice{\!\!\!}{\!\!}{\!\!}{\!\!}}%
\def\plaincdots@{\mathinner{\cdotp\cdotp\cdotp}}%
\def\intdots@{\mathchoice{\plaincdots@}%
 {{\cdotp}\mkern1.5mu{\cdotp}\mkern1.5mu{\cdotp}}%
 {{\cdotp}\mkern1mu{\cdotp}\mkern1mu{\cdotp}}%
 {{\cdotp}\mkern1mu{\cdotp}\mkern1mu{\cdotp}}}%
%
%
%  These macros are for doing the AMS \text{} construct
%
\def\RIfM@{\relax\protect\ifmmode}
\def\text{\RIfM@\expandafter\text@\else\expandafter\mbox\fi}
\let\nfss@text\text
\def\text@#1{\mathchoice
   {\textdef@\displaystyle\f@size{#1}}%
   {\textdef@\textstyle\tf@size{\firstchoice@false #1}}%
   {\textdef@\textstyle\sf@size{\firstchoice@false #1}}%
   {\textdef@\textstyle \ssf@size{\firstchoice@false #1}}%
   \glb@settings}

\def\textdef@#1#2#3{\hbox{{%
                    \everymath{#1}%
                    \let\f@size#2\selectfont
                    #3}}}
\newif\iffirstchoice@
\firstchoice@true
%
%These are the AMS constructs for multiline limits.
%
\def\Let@{\relax\iffalse{\fi\let\\=\cr\iffalse}\fi}%
\def\vspace@{\def\vspace##1{\crcr\noalign{\vskip##1\relax}}}%
\def\multilimits@{\bgroup\vspace@\Let@
 \baselineskip\fontdimen10 \scriptfont\tw@
 \advance\baselineskip\fontdimen12 \scriptfont\tw@
 \lineskip\thr@@\fontdimen8 \scriptfont\thr@@
 \lineskiplimit\lineskip
 \vbox\bgroup\ialign\bgroup\hfil$\m@th\scriptstyle{##}$\hfil\crcr}%
\def\Sb{_\multilimits@}%
\def\endSb{\crcr\egroup\egroup\egroup}%
\def\Sp{^\multilimits@}%
\let\endSp\endSb
%
%
%These are AMS constructs for horizontal arrows
%
\newdimen\ex@
\ex@.2326ex
\def\rightarrowfill@#1{$#1\m@th\mathord-\mkern-6mu\cleaders
 \hbox{$#1\mkern-2mu\mathord-\mkern-2mu$}\hfill
 \mkern-6mu\mathord\rightarrow$}%
\def\leftarrowfill@#1{$#1\m@th\mathord\leftarrow\mkern-6mu\cleaders
 \hbox{$#1\mkern-2mu\mathord-\mkern-2mu$}\hfill\mkern-6mu\mathord-$}%
\def\leftrightarrowfill@#1{$#1\m@th\mathord\leftarrow
\mkern-6mu\cleaders
 \hbox{$#1\mkern-2mu\mathord-\mkern-2mu$}\hfill
 \mkern-6mu\mathord\rightarrow$}%
\def\overrightarrow{\mathpalette\overrightarrow@}%
\def\overrightarrow@#1#2{\vbox{\ialign{##\crcr\rightarrowfill@#1\crcr
 \noalign{\kern-\ex@\nointerlineskip}$\m@th\hfil#1#2\hfil$\crcr}}}%
\let\overarrow\overrightarrow
\def\overleftarrow{\mathpalette\overleftarrow@}%
\def\overleftarrow@#1#2{\vbox{\ialign{##\crcr\leftarrowfill@#1\crcr
 \noalign{\kern-\ex@\nointerlineskip}$\m@th\hfil#1#2\hfil$\crcr}}}%
\def\overleftrightarrow{\mathpalette\overleftrightarrow@}%
\def\overleftrightarrow@#1#2{\vbox{\ialign{##\crcr
   \leftrightarrowfill@#1\crcr
 \noalign{\kern-\ex@\nointerlineskip}$\m@th\hfil#1#2\hfil$\crcr}}}%
\def\underrightarrow{\mathpalette\underrightarrow@}%
\def\underrightarrow@#1#2{\vtop{\ialign{##\crcr$\m@th\hfil#1#2\hfil
  $\crcr\noalign{\nointerlineskip}\rightarrowfill@#1\crcr}}}%
\let\underarrow\underrightarrow
\def\underleftarrow{\mathpalette\underleftarrow@}%
\def\underleftarrow@#1#2{\vtop{\ialign{##\crcr$\m@th\hfil#1#2\hfil
  $\crcr\noalign{\nointerlineskip}\leftarrowfill@#1\crcr}}}%
\def\underleftrightarrow{\mathpalette\underleftrightarrow@}%
\def\underleftrightarrow@#1#2{\vtop{\ialign{##\crcr$\m@th
  \hfil#1#2\hfil$\crcr
 \noalign{\nointerlineskip}\leftrightarrowfill@#1\crcr}}}%
%%%%%%%%%%%%%%%%%%%%%

\def\qopnamewl@#1{\mathop{\operator@font#1}\nlimits@}
\let\nlimits@\displaylimits
\def\setboxz@h{\setbox\z@\hbox}


\def\varlim@#1#2{\mathop{\vtop{\ialign{##\crcr
 \hfil$#1\m@th\operator@font lim$\hfil\crcr
 \noalign{\nointerlineskip}#2#1\crcr
 \noalign{\nointerlineskip\kern-\ex@}\crcr}}}}

 \def\rightarrowfill@#1{\m@th\setboxz@h{$#1-$}\ht\z@\z@
  $#1\copy\z@\mkern-6mu\cleaders
  \hbox{$#1\mkern-2mu\box\z@\mkern-2mu$}\hfill
  \mkern-6mu\mathord\rightarrow$}
\def\leftarrowfill@#1{\m@th\setboxz@h{$#1-$}\ht\z@\z@
  $#1\mathord\leftarrow\mkern-6mu\cleaders
  \hbox{$#1\mkern-2mu\copy\z@\mkern-2mu$}\hfill
  \mkern-6mu\box\z@$}


\def\projlim{\qopnamewl@{proj\,lim}}
\def\injlim{\qopnamewl@{inj\,lim}}
\def\varinjlim{\mathpalette\varlim@\rightarrowfill@}
\def\varprojlim{\mathpalette\varlim@\leftarrowfill@}
\def\varliminf{\mathpalette\varliminf@{}}
\def\varliminf@#1{\mathop{\underline{\vrule\@depth.2\ex@\@width\z@
   \hbox{$#1\m@th\operator@font lim$}}}}
\def\varlimsup{\mathpalette\varlimsup@{}}
\def\varlimsup@#1{\mathop{\overline
  {\hbox{$#1\m@th\operator@font lim$}}}}

%
%Companion to stackrel
\def\stackunder#1#2{\mathrel{\mathop{#2}\limits_{#1}}}%
%
%
% These are AMS environments that will be defined to
% be verbatims if amstex has not actually been 
% loaded
%
%
\begingroup \catcode `|=0 \catcode `[= 1
\catcode`]=2 \catcode `\{=12 \catcode `\}=12
\catcode`\\=12 
|gdef|@alignverbatim#1\end{align}[#1|end[align]]
|gdef|@salignverbatim#1\end{align*}[#1|end[align*]]

|gdef|@alignatverbatim#1\end{alignat}[#1|end[alignat]]
|gdef|@salignatverbatim#1\end{alignat*}[#1|end[alignat*]]

|gdef|@xalignatverbatim#1\end{xalignat}[#1|end[xalignat]]
|gdef|@sxalignatverbatim#1\end{xalignat*}[#1|end[xalignat*]]

|gdef|@gatherverbatim#1\end{gather}[#1|end[gather]]
|gdef|@sgatherverbatim#1\end{gather*}[#1|end[gather*]]

|gdef|@gatherverbatim#1\end{gather}[#1|end[gather]]
|gdef|@sgatherverbatim#1\end{gather*}[#1|end[gather*]]


|gdef|@multilineverbatim#1\end{multiline}[#1|end[multiline]]
|gdef|@smultilineverbatim#1\end{multiline*}[#1|end[multiline*]]

|gdef|@arraxverbatim#1\end{arrax}[#1|end[arrax]]
|gdef|@sarraxverbatim#1\end{arrax*}[#1|end[arrax*]]

|gdef|@tabulaxverbatim#1\end{tabulax}[#1|end[tabulax]]
|gdef|@stabulaxverbatim#1\end{tabulax*}[#1|end[tabulax*]]


|endgroup
  

  
\def\align{\@verbatim \frenchspacing\@vobeyspaces \@alignverbatim
You are using the "align" environment in a style in which it is not defined.}
\let\endalign=\endtrivlist
 
\@namedef{align*}{\@verbatim\@salignverbatim
You are using the "align*" environment in a style in which it is not defined.}
\expandafter\let\csname endalign*\endcsname =\endtrivlist




\def\alignat{\@verbatim \frenchspacing\@vobeyspaces \@alignatverbatim
You are using the "alignat" environment in a style in which it is not defined.}
\let\endalignat=\endtrivlist
 
\@namedef{alignat*}{\@verbatim\@salignatverbatim
You are using the "alignat*" environment in a style in which it is not defined.}
\expandafter\let\csname endalignat*\endcsname =\endtrivlist




\def\xalignat{\@verbatim \frenchspacing\@vobeyspaces \@xalignatverbatim
You are using the "xalignat" environment in a style in which it is not defined.}
\let\endxalignat=\endtrivlist
 
\@namedef{xalignat*}{\@verbatim\@sxalignatverbatim
You are using the "xalignat*" environment in a style in which it is not defined.}
\expandafter\let\csname endxalignat*\endcsname =\endtrivlist




\def\gather{\@verbatim \frenchspacing\@vobeyspaces \@gatherverbatim
You are using the "gather" environment in a style in which it is not defined.}
\let\endgather=\endtrivlist
 
\@namedef{gather*}{\@verbatim\@sgatherverbatim
You are using the "gather*" environment in a style in which it is not defined.}
\expandafter\let\csname endgather*\endcsname =\endtrivlist


\def\multiline{\@verbatim \frenchspacing\@vobeyspaces \@multilineverbatim
You are using the "multiline" environment in a style in which it is not defined.}
\let\endmultiline=\endtrivlist
 
\@namedef{multiline*}{\@verbatim\@smultilineverbatim
You are using the "multiline*" environment in a style in which it is not defined.}
\expandafter\let\csname endmultiline*\endcsname =\endtrivlist


\def\arrax{\@verbatim \frenchspacing\@vobeyspaces \@arraxverbatim
You are using a type of "array" construct that is only allowed in AmS-LaTeX.}
\let\endarrax=\endtrivlist

\def\tabulax{\@verbatim \frenchspacing\@vobeyspaces \@tabulaxverbatim
You are using a type of "tabular" construct that is only allowed in AmS-LaTeX.}
\let\endtabulax=\endtrivlist

 
\@namedef{arrax*}{\@verbatim\@sarraxverbatim
You are using a type of "array*" construct that is only allowed in AmS-LaTeX.}
\expandafter\let\csname endarrax*\endcsname =\endtrivlist

\@namedef{tabulax*}{\@verbatim\@stabulaxverbatim
You are using a type of "tabular*" construct that is only allowed in AmS-LaTeX.}
\expandafter\let\csname endtabulax*\endcsname =\endtrivlist

% macro to simulate ams tag construct


% This macro is a fix to the equation environment
 \def\endequation{%
     \ifmmode\ifinner % FLEQN hack
      \iftag@
        \addtocounter{equation}{-1} % undo the increment made in the begin part
        $\hfil
           \displaywidth\linewidth\@taggnum\egroup \endtrivlist
        \global\tag@false
        \global\@ignoretrue   
      \else
        $\hfil
           \displaywidth\linewidth\@eqnnum\egroup \endtrivlist
        \global\tag@false
        \global\@ignoretrue 
      \fi
     \else   
      \iftag@
        \addtocounter{equation}{-1} % undo the increment made in the begin part
        \eqno \hbox{\@taggnum}
        \global\tag@false%
        $$\global\@ignoretrue
      \else
        \eqno \hbox{\@eqnnum}% $$ BRACE MATCHING HACK
        $$\global\@ignoretrue
      \fi
     \fi\fi
 } 

 \newif\iftag@ \tag@false
 
 \def\TCItag{\@ifnextchar*{\@TCItagstar}{\@TCItag}}
 \def\@TCItag#1{%
     \global\tag@true
     \global\def\@taggnum{(#1)}}
 \def\@TCItagstar*#1{%
     \global\tag@true
     \global\def\@taggnum{#1}}

  \@ifundefined{tag}{
     \def\tag{\@ifnextchar*{\@tagstar}{\@tag}}
     \def\@tag#1{%
         \global\tag@true
         \global\def\@taggnum{(#1)}}
     \def\@tagstar*#1{%
         \global\tag@true
         \global\def\@taggnum{#1}}
  }{}
% Do not add anything to the end of this file.  
% The last section of the file is loaded only if 
% amstex has not been.



\makeatother
\endinput


\oddsidemargin 0in
\topmargin -0.5in
\textheight 9in
\textwidth 6.5in

\begin{document}

\title{A new development platform for parameter continuation and bifurcation
analysis in nonlinear dynamical systems}
\author{Harry Dankowicz$^{1}$ and Frank Schilder$^{2}$ \\
%EndAName
$^{1}$Department of Mechanical Science and Engineering \\
University of Illinois at Urbana-Champaign \\
Urbana, IL 61801, USA \\
danko@illinois.edu \\
$^{2}$Department of Mathematics \\
University of Surrey \\
Guildford, GU2 7XH, United Kingdom \\
f.schilder@surrey.ac.uk}
\maketitle

\begin{abstract}
\textbf{Key Words:} Bifurcation analysis, Parameter continuation, Software
development, \textsc{coco}
\end{abstract}

\section{Introduction}

A combination of theoretical and computational tools for bifurcation
analysis of dynamical systems offers distinct advantages to brute-force
forward-time simulation [9]. Such a combination enables prediction of
behavior and response without the need for a vast collection of simulations
based at distinct initial conditions. More importantly, it may offer an
understanding of, and an underlying explanation for, changes in behavior and
response that are not available through simple simulation.

A comprehensive bifurcation analysis of a dynamical system seeks to
establish the existence of characteristic classes of responses, such as
equilibria or periodic responses. In each case, this involves locating and
tracking families of such responses under variations in system parameters in
a process known as \emph{continuation} [1,5,8,10]. The study of the
robustness of particular system responses further emphasizes parameter
values where such families merge or terminate or where the stability
characteristics of the corresponding responses change.

A number of computational tools are available for bifurcation analysis of
characteristic classes of response, such as equilibria, periodic
trajectories, homo- or heteroclinic trajectories between equilibria and/or
periodic trajectories, quasiperiodic trajectories on invariant tori, and
trajectories on associated stable and unstable manifolds. These include
general algebraic and two-point boundary-value solvers for ordinary
differential equations, such as \textsc{auto} (and specialized drivers, such
as \textsc{homcont} [2], \textsc{slidecont} [3], and \textsc{tc-hat} [13]), 
\textsc{matcont} [4], and \textsc{sympercon} [14]; boundary-value solvers
for delay differential equations, such as \textsc{dde-biftool} [6] and 
\textsc{pdde-cont} [12]; tools for large-scale systems, such as \textsc{loca}
[11]; and implementations in \textsc{matlab} [7].

This presentation outlines a recent software development effort in \textsc{%
matlab} of a set of core toolboxes for parameter continuation of algebraic
and multi-point boundary-value problems referred to collectively as \textsc{%
coco}. In contrast with the packages referenced above, \textsc{coco} has
been formulated with emphasis on full transparency, modularity, and great
generality giving developers of bifurcation packages that rely on this
platform great flexibility and full access to the entirety of continuation
data. In addition, \textsc{coco} makes extensive use of vectorization and
bandwidth reduction for sparse-matrix operations to guarantee an optimal
execution time at or close to what can be expected of interpreted code.
Finally, \textsc{coco} includes a unique implementation of user-defined
functions that allows the continuation of solutions with nontrivial
properties, for example, periodic orbits with a given stability margin. The
presentation discusses the philosophy behind the software development, the
strategy employed in its implementation, and a number of model examples,
including continuation of periodic orbits in hybrid dynamical systems.

This material is based upon work supported by the National Science
Foundation under Grant nos. 0237370 (HD) and 0635469 (HD) and by the
Engineering and Physical Sciences Research Council under Grant no.
EP/D063906/1 (FS).

\section{Mathematical foundation}

The core of the \textsc{coco} application consists of a \emph{constructor}
that is employed to formulate a well-posed system of nonlinear algebraic
equations, an iterative \emph{corrector} that is employed to locate
individual points on the solution manifold, and a \emph{continuer} that is
employed to generate an approximate covering mesh for the solution manifold
during parameter continuation. The versatility of the \textsc{coco}
application lies in the generality with which variables, parameters, and
additional problem- or user-defined monitor functions are handled by the 
\textsc{coco} constructor.

\subsection{Illustration}

Consider the functions%
\[
\psi \left( x,y,z\right) =z^{2}-x^{2}-y^{2}
\]%
and%
\[
\pi \left( y,z,\alpha \right) =y\sin \alpha +\left( z-1\right) \cos \alpha 
\]%
in terms of the variables $x$, $y$, $z$, and $\alpha $. Roots of $\psi $
correspond to points in the three-dimensional space parametrized by $x$, $y$%
, and $z$ on the double circular cone with apex at the origin and symmetric
about the $z$-axis with a $90^{\circ }$ opening angle. For each $\alpha $, $%
\pi $ evaluates to the dot product between the unit vector $\left( 0,\sin
\alpha ,\cos \alpha \right) $ and the position vector from the point $\left(
0,0,1\right) $ to the point $\left( x,y,z\right) $, such that $\left| \pi
\left( y,z,\alpha \right) \right| $ equals the distance from the point $%
\left( x,y,z\right) $ to the plane through the point $\left( 0,0,1\right) $
and perpendicular to the vector $\left( 0,\sin \alpha ,\cos \alpha \right) $%
. Simultaneous roots of $\psi $ and $\pi $ thus correspond to points of
intersection of the cone and the corresponding plane (cf. Figure 1).\FRAME{%
fhFU}{2.6654in}{3.039in}{0pt}{\Qcb{Simultaneous roots of $\protect\psi $ and 
$\protect\pi $ correspond to poits of intersectio of the cone and the
corresponding plane.}}{}{cone.jpg}{\special{language "Scientific Word";type
"GRAPHIC";maintain-aspect-ratio TRUE;display "ICON";valid_file "F";width
2.6654in;height 3.039in;depth 0pt;original-width 4.6994in;original-height
5.3662in;cropleft "0.1202909";croptop "0.9457805";cropright
"0.9673136";cropbottom "0.0986235";filename 'figs/cone.jpg';file-properties
"XNPEU";}}

Consider the following tasks:

\begin{description}
\item[Task I.] \emph{Given an initial approximation }$\left( x^{\ast
},y^{\ast },z^{\ast }\right) $\emph{\ for a point on the cone, locate a
nearby point }$\left( x,y,z\right) \approx \left( x^{\ast },y^{\ast
},z^{\ast }\right) $\emph{\ on the cone with desired accuracy. }To allow for
the application of an iterative root-solver, the equation $\psi \left(
x,y,z\right) =0$ must be suitably augmented to ensure a non-singular
linearization at the sought-after solution. At the very least, the number of
equations $\dim _{eq}$ must equal the number of variables $\dim _{var}=3$,
here necessitating the imposition of two additional constraints on the
variables $x$, $y$, and $z$. As an example, the linearization of the function%
\begin{equation}
F\left( x,y,z\right) =\left( 
\begin{array}{c}
\psi \left( x,y,z\right)  \\ 
x-x^{\ast } \\ 
y-y^{\ast }%
\end{array}%
\right)   \label{eq: task i}
\end{equation}%
is non-singular provided that $z\neq 0$ at the sought-after solution. A
root-solver may now be employed to determine a root of $F$ corresponding to
a point on the cone with $x=x^{\ast }$, $y=y^{\ast }$ and $z\approx z^{\ast }
$.

\item[Task II.] \emph{Given an initial approximation }$\left( x^{\ast
},y^{\ast },z^{\ast }\right) $\emph{\ for a point of intersection of the
cone and the plane given by }$\alpha =\alpha ^{\ast }$\emph{, locate a
nearby point }$\left( x,y,z\right) \approx \left( x^{\ast },y^{\ast
},z^{\ast }\right) $\emph{\ of intersection of the cone and the plane given
by a nearby }$\alpha \approx \alpha ^{\ast }$\emph{.} Since, here, $\dim
_{var}=4$, it follows that the equations $\psi \left( x,y,z\right) =\pi
\left( y,z,\alpha \right) =0$ must be supplemented by the imposition of two
additional constraints on the variables $x$, $y$, $z$, and $\alpha $. As an
example, the linearization of the function%
\begin{equation}
F\left( x,y,z,\alpha \right) =\left( 
\begin{array}{c}
\psi \left( x,y,z\right)  \\ 
\pi \left( y,z,\alpha \right)  \\ 
x-x^{\ast } \\ 
\alpha -\alpha ^{\ast }%
\end{array}%
\right)   \label{eq: task ii a)}
\end{equation}%
is non-singular provided that $y\cos \alpha +z\sin \alpha \neq 0$ at the
sought-after solution. A root-solver may now be employed to determined a
root of $F$ corresponding to a point of intersection of the cone with $%
x=x^{\ast }$, $y\approx y^{\ast }$, and $z\approx z^{\ast }$, and the plane
given by $\alpha =\alpha ^{\ast }$. Alternatively, the linearization of the
function%
\begin{equation}
F\left( x,y,z,\alpha \right) =\left( 
\begin{array}{c}
\psi \left( x,y,z\right)  \\ 
\pi \left( y,z,\alpha \right)  \\ 
x-x^{\ast } \\ 
y-y^{\ast }%
\end{array}%
\right)   \label{eq: task ii b)}
\end{equation}%
is non-singular provided that $z,y\cos \alpha +\left( 1-z\right) \sin \alpha
\neq 0$ at the sought-after solution. A root-solver may now be employed to
determined a root of $F$ corresponding to a point of intersection of the
cone with $x=x^{\ast }$, $y=y^{\ast }$, and $z\approx z^{\ast }$, and the
plane given by $\alpha \approx \alpha ^{\ast }$.
\end{description}

Inspired by the above discussion, a fundamental objective of the \textsc{coco%
} constructor is to establish a general-purpose function $F$ that can be
suitably restricted to solve each of the tasks posed above. To this end,
consider the function%
\[
\tilde{F}\left( x,y,z,\alpha ,p_{0},p_{1},p_{2},p_{3}\right) =\left( 
\begin{array}{c}
\psi \left( x,y,z\right)  \\ 
\pi \left( y,z,\alpha \right) -p_{0} \\ 
x-p_{1} \\ 
y-p_{2} \\ 
\alpha -p_{3}%
\end{array}%
\right) .
\]%
The restriction 
\[
F=\left. \tilde{F}\right| _{p_{1}=x^{\ast },p_{2}=y^{\ast },p_{3}=\alpha
^{\ast }}
\]%
for some $\alpha ^{\ast }$ is then equivalent to Eq. (\ref{eq: task i}). In
particular, the linearization of $F$ is again non-singular provided that $%
z\neq 0$ at the sought-after solution. A root-solver may now be employed to
determine a root of $F$ corresponding to a value $p_{0}=\pi \left(
y,z,\alpha \right) $ where $\alpha =\alpha ^{\ast }$ and a point on the cone
with $x=x^{\ast }$, $y=y^{\ast }$, and $z\approx z^{\ast }$. In this case,
the value of $\pi $ at the sought-after solution is a by-product of the
root-finding algorithm.

Similarly, the restrictions 
\[
F_{1}=\left. \tilde{F}\right| _{p_{0}=0,p_{1}=x^{\ast },p_{3}=\alpha ^{\ast
}}
\]%
and 
\[
F_{2}=\left. \tilde{F}\right| _{p_{0}=0,p_{1}=x^{\ast },p_{2}=y^{\ast }}
\]%
are equivalent to Eqs. (\ref{eq: task ii a)}) and (\ref{eq: task ii b)}),
respectively. In particular, the linearizations of $F_{1}$ and $F_{2}$ are
again non-singular provided that $y\cos \alpha +z\sin \alpha \neq 0$ or $%
z,y\cos \alpha +\left( 1-z\right) \sin \alpha \neq 0$, respectively, at the
sought-after solution. Finally, a root-solver may be employed to determine a
root of $F_{1}$ corresponding to a value $p_{2}=y$ and a point of
intersection of the cone with $x=x^{\ast }$, $y\approx y^{\ast }$, and $%
z\approx z^{\ast }$ and the plane given by $\alpha =\alpha ^{\ast }$.
Alternatively, a root-solver may be employed to determine a root of $F_{2}$
corresponding to a value $p_{3}=\alpha $ and a point of intersection of the
cone with $x=x^{\ast }$, $y=y^{\ast }$, and $z\approx z^{\ast }$ and the
plane given by $\alpha \approx \alpha ^{\ast }$, respectively.

\subsection{Generalizations}

\textsc{coco} generalizes the formalism of the previous section to arbitrary
problem and variable dimension. To this end, let $\xi \in \mathbb{R}^{n}$
and $\rho \in \mathbb{R}^{m}$ and denote by $f:\mathbb{R}^{n}\times \mathbb{R%
}^{m}\rightarrow \mathbb{R}^{n}$ and $g:\mathbb{R}^{n}\times \mathbb{R}%
^{m}\rightarrow \mathbb{R}^{k}$ two sufficiently smooth vector-valued
functions. Let $\tilde{g}:\mathbb{R}^{n}\times \mathbb{R}^{m}\rightarrow 
\mathbb{R}^{k+m}$ be given by%
\[
\tilde{g}\left( \xi ,\rho \right) =\left( 
\begin{array}{c}
g\left( \xi ,\rho \right)  \\ 
\rho 
\end{array}%
\right) 
\]%
and consider the function $\tilde{F}:\mathbb{R}^{n}\times \mathbb{R}%
^{m}\times \mathbb{R}^{k+m}\rightarrow \mathbb{R}^{n+k+m}$, where%
\[
\tilde{F}\left( \xi ,\rho ,\tilde{\eta}\right) =\left( 
\begin{array}{c}
f\left( \xi ,\rho \right)  \\ 
\tilde{g}\left( \xi ,\rho \right) -\tilde{\eta}%
\end{array}%
\right) 
\]%
with rectangular Jacobian%
\[
\partial _{\left( \xi ,\rho ,\tilde{\eta}\right) }\tilde{F}\left( \xi ,\rho ,%
\tilde{\eta}\right) =\left( 
\begin{array}{ccc}
\partial _{\xi }f\left( \xi ,\rho \right)  & \partial _{\rho }f\left( \xi
,\rho \right)  & 0 \\ 
\partial _{\xi }\tilde{g}\left( \xi ,\rho \right)  & \partial _{\rho }\tilde{%
g}\left( \xi ,\rho \right)  & -I_{k+m}%
\end{array}%
\right) 
\]%
where $I_{k+m}$ denotes the $\left( k+m\right) \times \left( k+m\right) $
identity matrix and $0$ the (appropriately dimensioned) null matrix.

\subsubsection{Initialization}

Suppose that there exists a combination $\xi =\xi ^{\ast }$ and $\rho =\rho
^{\ast }$ such that $f\left( \xi ^{\ast },\rho ^{\ast }\right) =0$, and let $%
\tilde{\eta}^{\ast }=\tilde{g}\left( \xi ^{\ast },\rho ^{\ast }\right) $.
Denote by $\mathcal{M}$ the co-dimension-$m$ manifold through $\tilde{\eta}^{\ast }$ obtained by fixing $m$ components of $\tilde{\eta}$ and trivially
parametrized by $\xi $, $\rho $, and the remaining $k$ components $\eta $ of 
$\tilde{\eta}$. Suppose that the linearization of the restriction $F=\left. 
\tilde{F}\right| _{\mathcal{M}}$ at $\xi =\xi ^{\ast }$, $\rho =\rho ^{\ast
} $, and $\eta =\eta ^{\ast }$ is non-singular. It follows from the
inverse-function theorem that there exist neighborhoods $U\ni $ $\left( 
\begin{array}{ccc}
\xi ^{\ast } & \rho ^{\ast } & \eta ^{\ast }
\end{array}
\right) ^{T}$ and $V\ni 0$, such that $F:U\rightarrow V$ is one-to-one and
onto and, consequently, that $\xi =\xi ^{\ast }$ and $\rho =\rho ^{\ast }$
is a locally unique root for the \emph{zero-problem} $f$ with $\tilde{\eta}
^{\ast }=\tilde{g}\left( \xi ^{\ast },\rho ^{\ast }\right) $.

Alternatively, suppose that $f\left( \xi ^{\left( 0\right) },\rho ^{\left(
0\right) }\right) \approx 0$ and $\tilde{\eta}^{\left( 0\right) }\approx 
\tilde{g}\left( \xi ^{\left( 0\right) },\rho ^{\left( 0\right) }\right) $
for some $\xi ^{\left( 0\right) }$, $\rho ^{\left( 0\right) }$, and $\tilde{
\eta}^{\left( 0\right) }$ and denote by $\mathcal{M}$ the co-dimension-$m$
manifold through $\tilde{\eta}^{\left( 0\right) }$ obtained by fixing $m$
components of $\tilde{\eta}$ and trivially parametrized by $\xi $, $\rho $,
and the remaining $k$ components $\eta $ of $\tilde{\eta}$. Provided that
there exists a nearby root of $F=\left. \tilde{F}\right| _{\mathcal{M}}$ at
which the linearization is non-singular, one step of an iterative
root-solver may now be applied to $F$ to yield an improved approximation $
\xi ^{\left( 1\right) }$, $\rho ^{\left( 1\right) }$, and $\eta ^{\left(
1\right) }$.

\begin{example}
In the notation of the previous section, let $\xi =z$, $\rho =\left( 
\begin{array}{ccc}
x & y & \alpha 
\end{array}%
\right) ^{T}$, $f\left( \xi ,\rho \right) =\psi \left( x,y,z\right) $, and $%
g\left( \xi ,\rho \right) =\pi \left( y,z,\alpha \right) $ and suppose that $%
\psi \left( x^{\ast },y^{\ast },z^{\ast }\right) =0$ for some $x^{\ast }$, $%
y^{\ast }$, and $z^{\ast }$. Then, for some $\alpha ^{\ast }$, 
\[
F\left( \xi ,\rho ,\eta \right) =\left. \tilde{F}\left( \xi ,\rho ,\tilde{%
\eta}\right) \right| _{\tilde{\eta}=\left( \eta ,x^{\ast },y^{\ast },\alpha
^{\ast }\right) }=\left( 
\begin{array}{c}
\psi \left( x,y,z\right)  \\ 
\pi \left( y,z,\alpha \right) -\eta  \\ 
x-x^{\ast } \\ 
y-y^{\ast } \\ 
\alpha -\alpha ^{\ast }%
\end{array}%
\right) \Rightarrow \left| \partial _{\left( \xi ,\rho ,\eta \right)
}F\left( \xi ^{\ast },\rho ^{\ast },\eta ^{\ast }\right) \right| =2z^{\ast }
\]%
and it follows from the previous section that, as long as $z^{\ast }\neq 0$,
the inverse-function theorem guarantees that $x=x^{\ast }$, $y=y^{\ast }$, $%
z=z^{\ast }$, $\alpha =\alpha ^{\ast }$, and $\eta =\eta ^{\ast }=\pi \left(
y^{\ast },z^{\ast },\alpha ^{\ast }\right) $ is a locally unique root of $F$.
\end{example}

\begin{example}
Suppose, instead, that $\psi \left( x^{\ast },y^{\ast },z^{\ast }\right)
=\pi \left( y^{\ast },z^{\ast },\alpha ^{\ast }\right) =0$ for some $x^{\ast
}$, $y^{\ast }$, $z^{\ast }$, and $\alpha ^{\ast }$. Then,%
\[
F\left( \xi ,\rho ,\eta \right) =\left. \tilde{F}\left( \xi ,\rho ,\tilde{%
\eta}\right) \right| _{\tilde{\eta}=\left( 0,x^{\ast },\eta ,\alpha ^{\ast
}\right) }=\left( 
\begin{array}{c}
\psi \left( x,y,z\right)  \\ 
\pi \left( y,z,\alpha \right)  \\ 
x-x^{\ast } \\ 
y-\eta  \\ 
\alpha -\alpha ^{\ast }%
\end{array}%
\right) \Rightarrow \left| \partial _{\left( \xi ,\rho ,\eta \right)
}F\left( \xi ^{\ast },\rho ^{\ast },\eta ^{\ast }\right) \right| =-2\left(
y^{\ast }\cos \alpha ^{\ast }+z^{\ast }\sin \alpha ^{\ast }\right) 
\]%
and it follows from the previous section that, as long as $y^{\ast }\cos
\alpha ^{\ast }+z^{\ast }\sin \alpha ^{\ast }\neq 0$, the inverse-function
theorem guarantees that $x=x^{\ast }$, $y=y^{\ast }$, $z=z^{\ast }$, $\alpha
=\alpha ^{\ast }$, and $\eta =\eta ^{\ast }=y^{\ast }$ is a locally unique
root of $F$.
\end{example}

In the special case that$\ \mathcal{M}$ is obtained by fixing the last $m$
components of $\tilde{\eta}$, the linearization of $F$ at $\xi =\xi ^{\ast }$%
, $\rho =\rho ^{\ast }$, and $\eta =\eta ^{\ast }$ is non-singular if and
only if the linearization of $\left. f\right| _{\rho =\rho ^{\ast }}$ at $%
\xi =\xi ^{\ast }$ is non-singular.

\subsubsection{Continuation}

Suppose again that there exists a combination $\xi =\xi ^{\ast }$ and $\rho
=\rho ^{\ast }$ such that $f\left( \xi ^{\ast },\rho ^{\ast }\right) =0$,
and let $\tilde{\eta}^{\ast }=\tilde{g}\left( \xi ^{\ast },\rho ^{\ast
}\right) $. Denote by $\mathcal{M}$ the co-dimension-$m-j$ manifold through $%
\tilde{\eta}^{\ast }$ obtained by fixing $m-j$ components of $\tilde{\eta}$
and trivally parametrized by $\xi $, $\rho $, and the remaining $k+j$
components $\eta $ of $\tilde{\eta}$. Suppose that there exists a function $%
h:\mathbb{R}^{n}\times \mathbb{R}^{m}\times \mathbb{R}^{k+j}\times \mathbb{R}%
^{j}\rightarrow \mathbb{R}^{j}$, such that $h\left( \xi ^{\ast },\rho ^{\ast
},\eta ^{\ast },0\right) =0$ and such that the linearization $\partial
_{\left( \xi ,\rho ,\eta \right) }F\left( \xi ,\rho ,\eta ,\lambda \right) $
of the function%
\begin{equation}
F\left( \xi ,\rho ,\eta ,\lambda \right) =\left( 
\begin{array}{c}
\left. \tilde{F}\right| _{\mathcal{M}}\left( \xi ,\rho ,\eta \right) \\ 
h\left( \xi ,\rho ,\eta ,\lambda \right)%
\end{array}%
\right)  \label{eq: Continuation}
\end{equation}%
is non-singular at $\xi =\xi ^{\ast }$, $\rho =\rho ^{\ast }$, $\eta =\eta
^{\ast }$, and $\lambda =0$. Since 
\[
F\left( \xi ^{\ast },\rho ^{\ast },\eta ^{\ast },0\right) =0, 
\]%
it follows by the implicit-function theorem that there exists locally unique
functions $\xi \left( \lambda \right) $, $\rho \left( \lambda \right) $, and 
$\eta \left( \lambda \right) $ for $\lambda \approx 0$, such that $\xi
\left( 0\right) =\xi ^{\ast }$, $\rho \left( 0\right) =\rho ^{\ast }$, and $%
\eta \left( 0\right) =\eta ^{\ast }$ and%
\[
F\left( \xi \left( \lambda \right) ,\rho \left( \lambda \right) ,\eta \left(
\lambda \right) ,\lambda \right) =0. 
\]

In particular, suppose that%
\begin{equation}
h\left( \xi ,\rho ,\eta ,\lambda \right) =V^{T}\cdot \left( 
\begin{array}{c}
\xi -\xi ^{\ast } \\ 
\rho -\rho ^{\ast } \\ 
\eta -\eta ^{\ast }%
\end{array}%
\right) -\lambda   \label{eq: hlinear}
\end{equation}%
satisfies the non-singularity condition for some rank-$j$ matrix $V$. It
follows that the point $\left( \xi \left( \lambda \right) ,\rho \left(
\lambda \right) ,\eta \left( \lambda \right) \right) $ lies at the
intersection of $j$ hyperplanes determined by the components of $\lambda $
and with normal vectors given by the columns of $V$. Moreover, for $\lambda
\approx 0$, the columns of the matrix%
\[
W=\left( \partial _{\left( \xi ,\rho ,\eta \right) }F\left( \xi \left(
\lambda \right) ,\rho \left( \lambda \right) ,\eta \left( \lambda \right)
,\lambda \right) \right) ^{-1}\cdot \left( 
\begin{array}{c}
0 \\ 
I_{j}%
\end{array}%
\right) 
\]%
span the $j$-dimensional nullspace of the linearization of $\left. \tilde{F}%
\right| _{\mathcal{M}}$ at $\xi =\xi \left( \lambda \right) $, $\rho =\rho
\left( \lambda \right) $, and $\eta =\eta \left( \lambda \right) $.

As a special case, choose $V$ such that%
\[
h\left( \xi ,\rho ,\eta ,\lambda \right) =\hat{\eta}-\hat{\eta}^{\ast
}-\lambda ,
\]%
where $\hat{\eta}$ denotes some subset of $j$ components of $\eta $. It
follows that $\hat{\eta}\left( \lambda \right) =\hat{\eta}^{\ast }+\lambda $
and the case when $\lambda =0$ is equivalent to the task of initialization
described previously. Alternatively, suppose that the $j$ columns of the
matrix $V$ are a basis for the nullspace of the linearization of the
restriction $\left. \tilde{F}\right| _{\mathcal{M}}$ at $\xi =\xi ^{\ast }$, 
$\rho =\rho ^{\ast }$, and $\eta =\eta ^{\ast }$. In this case, the $j$%
-dimensional solution manifold of the equation $\left. \tilde{F}\right| _{%
\mathcal{M}}\left( \xi ,\rho ,\eta \right) =0$ is locally represented as a
graph over the tangent space at $\left( \xi ^{\ast },\rho ^{\ast },\eta
^{\ast }\right) $ parametrized by $\lambda $.

\begin{example}
\label{y continuation}In the notation of the previous section, let $\xi =z$, 
$\rho =\left( 
\begin{array}{ccc}
x & y & \alpha%
\end{array}%
\right) ^{T}$, $f\left( \xi ,\rho \right) =\psi \left( x,y,z\right) $, and $%
g\left( \xi ,\rho \right) =\pi \left( y,z,\alpha \right) $ and suppose that $%
\psi \left( x^{\ast },y^{\ast },z^{\ast }\right) =0$ for some $x^{\ast }$, $%
y^{\ast }$, and $z^{\ast }$. Then, for some $\alpha ^{\ast }$ and provided
that $z^{\ast }\neq 0$, the nullspace of the linearization of the
restriction $\left. \tilde{F}\left( \xi ,\rho ,\tilde{\eta}\right) \right| _{%
\tilde{\eta}=\left( \eta _{0},x^{\ast },\eta _{1},\alpha ^{\ast }\right) }$
at $\xi =\xi ^{\ast }$, $\rho =\rho ^{\ast }$, and $\eta =\eta ^{\ast }$ is
spanned by the column matrix%
\[
\left( 
\begin{array}{c}
y^{\ast } \\ 
0 \\ 
z^{\ast } \\ 
0 \\ 
y^{\ast }\cos \alpha ^{\ast }+z^{\ast }\sin \alpha ^{\ast } \\ 
z^{\ast }%
\end{array}%
\right) . 
\]%
The implicit-function theorem guarantees the existence of a locally unique
root of the augmented vector-valued function in Eq. (\ref{eq: Continuation})
for each value of $\lambda \approx 0$, where $\mathcal{M}=\left\{ \tilde{\eta%
}=\left( \eta _{0},x^{\ast },\eta _{1},\alpha ^{\ast }\right) \right\} $.
Given an initial prediction $x\approx x^{\ast }$, $y\approx y^{\ast }$, $%
z\approx z^{\ast }$, $\alpha \approx \alpha ^{\ast }$, $\eta _{0}\approx \pi
\left( y^{\ast },z^{\ast },\alpha ^{\ast }\right) $, and $\eta _{1}\approx
y^{\ast }$ for some $\lambda \approx 0$, an iterative root solver may now be
employed to locate the corresponding root including $\eta _{0}\left( \lambda
\right) =\pi \left( y\left( \lambda \right) ,z\left( \lambda \right) ,\alpha
\left( \lambda \right) \right) $.
\end{example}

\begin{example}
\label{alpha and y continuation}Suppose, instead, that $\psi \left( x^{\ast
},y^{\ast },z^{\ast }\right) =\pi \left( y^{\ast },z^{\ast },\alpha ^{\ast
}\right) =0$ for some $x^{\ast }$, $y^{\ast }$, $z^{\ast }$, and $\alpha
^{\ast }$. Then, unless $x^{\ast 2}-1=y^{\ast }=z^{\ast }-1=\sin \alpha
^{\ast }=0$, the nullspace of the linearization of the restriction $\left. 
\tilde{F}\left( \xi ,\rho ,\tilde{\eta}\right) \right| _{\tilde{\eta}=\left(
0,x^{\ast },\eta _{0},\eta _{1}\right) }$ at $\xi =\xi ^{\ast }$, $\rho
=\rho ^{\ast }$, and $\eta =\eta ^{\ast }$ is spanned by the column matrix%
\[
\left( 
\begin{array}{c}
y^{\ast }\left( y^{\ast }\cos \alpha ^{\ast }-\left( z^{\ast }-1\right) \sin
\alpha ^{\ast }\right) \\ 
0 \\ 
z^{\ast }\left( y^{\ast }\cos \alpha ^{\ast }-\left( z^{\ast }-1\right) \sin
\alpha ^{\ast }\right) \\ 
-y^{\ast }\cos \alpha ^{\ast }-z^{\ast }\sin \alpha ^{\ast } \\ 
z^{\ast }\left( y^{\ast }\cos \alpha ^{\ast }-\left( z^{\ast }-1\right) \sin
\alpha ^{\ast }\right) \\ 
-y^{\ast }\cos \alpha ^{\ast }-z^{\ast }\sin \alpha ^{\ast }%
\end{array}%
\right) . 
\]%
The implicit-function theorem guarantees the existence of a locally unique
root of the augmented vector-valued function in Eq. (\ref{eq: Continuation})
for each value of $\lambda \approx 0$, where $\mathcal{M}=\left\{ \tilde{\eta%
}=\left( 0,x^{\ast },\eta _{0},\eta _{1}\right) \right\} $. Given an initial
prediction $x\approx x^{\ast }$, $y\approx y^{\ast }$, $z\approx z^{\ast }$, 
$\alpha \approx \alpha ^{\ast }$, $\eta _{0}\approx y^{\ast }$, and $\eta
_{1}\approx \alpha ^{\ast }$ for some $\lambda \approx 0$, an iterative root
solver may now be employed to locate the corresponding root.\newline
\newline
Similarly, unless $x^{\ast }=y^{\ast }=z^{\ast }=\cos \alpha ^{\ast }=0$,
the nullspace of the linearization of the restriction $\left. \tilde{F}%
\left( \xi ,\rho ,\tilde{\eta}\right) \right| _{\tilde{\eta}=\left( 0,\eta
_{0},\eta _{1},\alpha ^{\ast }\right) }$ at $\xi =\xi ^{\ast }$, $\rho =\rho
^{\ast }$, and $\eta =\eta ^{\ast }$ is spanned by the column matrix%
\[
\left( 
\begin{array}{c}
x^{\ast }\tan \alpha ^{\ast } \\ 
y^{\ast }+z^{\ast }\tan \alpha ^{\ast } \\ 
-x^{\ast } \\ 
0 \\ 
y^{\ast }+z^{\ast }\tan \alpha ^{\ast } \\ 
-x^{\ast }%
\end{array}%
\right) 
\]%
The implicit-function theorem guarantees the existence of a locally unique
root of the augmented vector-valued function in Eq. (\ref{eq: Continuation})
for each value of $\lambda \approx 0$, where $\mathcal{M}=\left\{ \tilde{\eta%
}=\left( 0,\eta _{0},\eta _{1},\alpha ^{\ast }\right) \right\} $. Given an
initial prediction $x\approx x^{\ast }$, $y\approx y^{\ast }$, $z\approx
z^{\ast }$, $\alpha \approx \alpha ^{\ast }$, $\eta _{0}\approx x^{\ast }$,
and $\eta _{1}\approx y^{\ast }$ for some $\lambda \approx 0$, an iterative
root solver may now be employed to locate the corresponding root.
\end{example}

In the special case that $\mathcal{M}$ is obtained by fixing $m-j$ of the
last $m$ components of $\tilde{\eta}$, the linearization of $\left. \tilde{F}%
\right| _{\mathcal{M}}$ at $\xi =\xi ^{\ast }$, $\rho =\rho ^{\ast }$, and $%
\eta =\eta ^{\ast }$ has a $j$-dimensional nullspace if and only if the
matrix 
\[
\left( 
\begin{array}{cc}
\partial _{\xi }f\left( \xi ^{\ast },\rho ^{\ast }\right) & \partial _{\rho
_{0}}f\left( \xi ^{\ast },\rho ^{\ast }\right)%
\end{array}%
\right) 
\]
has a $j$-dimensional nullspace, where $\rho _{0}$ refers to the first $j$
components of $\rho $.

\subsection{Continuation with constraints}

Suppose again that there exists a combination $\xi =\xi ^{\ast }$, $\rho
=\rho ^{\ast }$ such that $f\left( \xi ^{\ast },\rho ^{\ast }\right) =0$ and
let $\tilde{\eta}^{\ast }=\tilde{g}\left( \xi ^{\ast },\rho ^{\ast }\right) $%
. There exist $\binom{k+m}{m-j}$ distinct co-dimension-$m-j$ manifolds $%
\mathcal{M}$ through $\tilde{\eta}^{\ast }$ corresponding to fixing the
value of $m-j$ components of $\tilde{\eta}$ and trivially parametrized by $%
\xi $, $\rho $, and the remaining $k+j$ components of $\tilde{\eta}$. If the
point $\left( \xi ^{\ast },\rho ^{\ast },\tilde{\eta}^{\ast }\right) $ has
been reached during continuation along one of these manifolds, continuation
can proceed along one of the other manifolds simply by redefining $\eta $ to
refer to the corresponding $k+j$ components of $\tilde{\eta}$. This manifold
switching allows one to fix or release values of individual components of $%
\tilde{g}$, and therefore of $g$ and $\rho $, respectively.

Suppose that, initially, continuation occurs along a manifold $\mathcal{M}$
obtained by fixing $m-j$ components of the last $m$ components of $\tilde{%
\eta}$. This corresponds to keeping the values of $m-j$ components of $\rho $
constant during continuation while locating roots of the zero problem $f$
and monitoring the values of the remaining components of $\tilde{g}$ through
the value of $\eta $. At any stage during continuation, manifold switching
could then occur by releasing one component of $\rho $ and fixing one of the
components of $g$, corresponding to co-dimension-one parameter continuation.
This corresponds to constraining the continuation to maintain a constant
value of the selected component of $g$. Repeated application of this
procedure is possible until all components of $\rho $ have been released
affording a means for at most co-dimension-$m-j$ continuation.

An example of this analysis is given by the transition between the two
examples above. In particular, in Example \ref{y continuation}, continuation
along the $\mathcal{M}=\left\{ \tilde{\eta}=\left( \eta _{0},x^{\ast },\eta
_{1},\alpha ^{\ast }\right) \right\} $ manifold generates a sequence of
points on the cone for which $x=x^{\ast }$. A zero-crossing of the value of $%
\eta _{0}$ then corresponds to a point of intersection of the cone and the
plane given by $\alpha =\alpha ^{\ast }$. In the second part of Example \ref%
{alpha and y continuation}, this serves as the starting point for a
continuation along the $\mathcal{M}=\left\{ \tilde{\eta}=\left( 0,\eta
_{0},\eta _{1},\alpha ^{\ast }\right) \right\} $ manifold that generates a
sequence of points of intersection between the cone and the plane given by $%
\alpha =\alpha ^{\ast }$ (cf. left panel of Figure 2 and Appendix B). \FRAME{%
fhFU}{4.9856in}{2.0332in}{0pt}{\Qcb{Left panel: the result of continuation
of points on the cone with $x=x^{\ast }$ under variations in $y$ (blue
dots); detection of points of intersection with planes corresponding to
distinct values of $\protect\alpha $ (red dots); and continuation of points
of intersection between the cone and plane under variations in $x$ and $y$
for each such value of $\protect\alpha $ (green dots). Right panel: the
result of continuation of points on the cone with $x=x^{\ast }$ under
variations in $y$ (blue dots); detection of a point of intersection with the
plane corresponding to $\protect\alpha ^{\ast }$ (green dot); continuation
of points on the plane corresponding to $\protect\alpha ^{\ast }$ with $%
x=x^{\ast }$ under variations in $y$ (black dots); detection of a point of
intersection with the cone (red dot); and continuation of points on the cone
with $x=x^{\ast }$ under variations in $y$ (blue dots).}}{}{cones.eps}{%
\special{language "Scientific Word";type "GRAPHIC";maintain-aspect-ratio
TRUE;display "ICON";valid_file "F";width 4.9856in;height 2.0332in;depth
0pt;original-width 4.9571in;original-height 2.0055in;cropleft "0";croptop
"1";cropright "1";cropbottom "0";filename 'figs/cones.eps';file-properties
"XNPEU";}}

Alternatively, suppose that $n=0$ and let $\rho =\left( 
\begin{array}{cccc}
z & x & y & \alpha 
\end{array}%
\right) ^{T}$ and $g\left( \xi ,\rho \right) =\left( 
\begin{array}{cc}
\psi \left( x,y,z\right)  & \pi \left( y,z,\alpha \right) 
\end{array}%
\right) ^{T}$. Continuation along the manifold $\mathcal{M}=\left\{ \tilde{%
\eta}=\left( 0,\eta _{0},\eta _{1},x^{\ast },\eta _{2},\alpha ^{\ast
}\right) \right\} $ then generates a sequence of points on the cone for
which $x=x^{\ast }$. Again, a zero-crossing of the value of $\eta _{0}$
corresponds to a point of intersection of the cone and the plane given by $%
\alpha =\alpha ^{\ast }$. This may serve as the starting point for a
subsequent continuation along the manifold $\mathcal{M}=\left\{ \tilde{\eta}%
=\left( \eta _{0},0,\eta _{1},x^{\ast },\eta _{2},\alpha ^{\ast }\right)
\right\} $ that generates a sequence of points on the plane with $\alpha
=\alpha ^{\ast }$ for which $x=x^{\ast }$ (cf. right panel of Figure 2 and
Appendix B).

As a more intriguing example, consider the problem of monitoring the
eigenvalues of the jacobian matrix $A=\partial _{\xi }f$, of detecting
zero-crossings of the real parts of these eigenvalues, and of
higher-codimension continuation of families of roots of $f$ corresponding to
such zero-crossings. For this purpose, augment $\xi $ to include the real
and imaginary entries of the matrix $Q$ and the nontrivial entries of the
upper triangular matrix $U$ and augment $f$ by the entries of the matrix%
\[
A-Q^{\ast }UQ 
\]%
and the upper triangular entries of the matrix%
\[
Q^{\ast }Q-I. 
\]%
The roots of the augmented $f$ then include the roots of the original
function $f$ as well as the entries in the Schur decomposition of $A$ in
terms of a unitary matrix $Q$ and an upper triangular matrix $U$. In
particular, the entries of $U$ along the diagonal are the eigenvalues of $A$
repeated according to their multiplicity.

As long as the eigenvalues of $A$ are distinct, the Schur decomposition is
unique up to reordering of the columns of $Q$ and a corresponding
permutation of the entries of $U$. For a given permutation, the real and
imaginary parts of the entries of $U$ are smooth functions of the system
parameters.\ In particular, the real parts of the diagonal entries may be
included in the function $g$ and their values may thus be monitored during
continuation. Subsequent to detecting a zero-crossing of one or several of
these real parts, the corresponding components of $g$ may be fixed, with a
corresponding release of components of $\rho $ as described above.

\section{Implementation}

Communication between \textsc{coco} components during execution is
accomplished through the passing and retrieval of a reference to the global 
\texttt{opts} structure. Class members of this structure serve to identify
properties that govern the execution of distinct \textsc{coco} components as
well as to store intermediate computational results. The discussion below
details the main elements of \textsc{coco} and the affected classes and
class fields within the \texttt{opts} structure.

\subsection{The \textsc{coco} constructor}

The \textsc{coco} constructor serves to define a restriction $\left. \tilde{F%
}\right| _{\mathcal{M}}$, a set of primary continuation parameters, and an
associated event structure for continuation and event handling using the 
\textsc{coco} continuer. Specifically, constructor functions enable the
specification of the ``zero-problem'' algorithm, a set of monitor functions
and associated events, and index vectors that determine the continuation
status of the monitor functions. Here, events refer to the (possibly
simultaneous) attainment of predefined numerical value(s) for a (collection
of) monitor function(s). Properties of the \texttt{efunc} class store the
relevant information.

\subsubsection{The zero-algorithm and monitor functions}

\begin{itemize}
\item Information regarding the algorithm $f:\mathbb{R}^{n}\times \mathbb{R}%
^{m}\rightarrow \mathbb{R}^{n}$ is provided to \textsc{coco} using the
function \texttt{coco\_set\_func}. In particular, provided that $n\neq 0$, a
reference to $f$ is assigned to \texttt{efunc.func} (which is otherwise
empty). Moreover, the integer $m$ is assigned to \texttt{efunc.func\_dim}.

\item Information regarding monitor functions and their default behavior
during continuation is provided to \textsc{coco} using the function \texttt{%
coco\_add\_func}. In particular, each call to \texttt{coco\_add\_func}%
\textbf{\ }appends to \texttt{efunc.idx2par} an array of string labels that
reference the components of a vector-valued monitor function and allocates
the corresponding indices depending on the default behavior of the monitor
function during continuation.

\item A monitor function is labeled \emph{inactive} if it should be included
in $g$ and if its default behavior is to remain constant during
continuation. The corresponding indices are appended to \texttt{%
efunc.inactive\_pars}. The value of \texttt{efunc.userA\_dim} (which is
initialized to $0$) is reduced by the range dimension of the corresponding
monitor function. Continuation is only possible as long as the total number
of elements of \texttt{efunc.inactive\_pars} equals the integer $m$.

\item A monitor function is labeled \emph{active} or \emph{internal }if it
should be included in $g$ and if its default behavior is to vary during
continuation and to be included in the \textsc{coco} continuer event
monitoring. The corresponding indices are appended to \texttt{%
efunc.active\_pars} or \texttt{efunc.internal\_pars}, respectively.

\item A monitor function is labeled \emph{regular} or \emph{singular} if it
should be omitted from $g$ and if its default behavior is to vary during
continuation and to be included in the \textsc{coco} continuer event
monitoring. Such monitor functions are regular provided that the
linearizations of (\ref{eq: Continuation}) at associated events are full
rank and singular, otherwise. The corresponding indices are appended to 
\texttt{efunc.regular\_pars} or \texttt{efunc.special\_pars}, respectively. 

\item A structure is appended to \texttt{efunc.userA} for each monitor
function with inactive, active, or internal default behavior; to \texttt{%
efunc.userB} for each monitor function with regular default behavior; and to 
\texttt{efunc.userC} for each monitor function with singular default
behavior. In each case, the structure's \texttt{idx} field contains a vector
of the corresponding indices, the structure's \texttt{F} field contains a
reference to the function, and the structure's \texttt{DFDX} field contains
a reference to the linearization of the function, when available.
\end{itemize}

\subsubsection{Continuation}

\begin{itemize}
\item Information regarding the actual continuation status of individual
components of monitor functions with inactive, active, or internal default
behavior is provided to \textsc{coco} using the functions \texttt{%
coco\_xchg\_pars} and \texttt{coco\_set\_dim}. In particular, each call to 
\texttt{coco\_xchg\_pars} appends to \texttt{efunc.xchg} a pair of string
labels corresponding to a transposition of the continuation status of the
associated monitor functions.

\item Each call to \texttt{coco\_set\_dim} assigns to \texttt{efunc.dim} the
dimension $j$ of the solution manifold; to \texttt{efunc.cont\_pars} the
labels of a subset of monitor functions; and to \texttt{efunc.cont\_par\_idx}
the associated indices. In particular, the first $j$ elements of \texttt{%
efunc.cont\_pars} must correspond to monitor functions with inactive
continuation status. For each subsequent element of \texttt{efunc.cont\_pars}%
, its continuation status is temporarily exchanged for that of a monitor
function with internal continuation status, if such a function exists. The
first $j$ elements of \texttt{efunc.cont\_par\_idx} and the indices of
monitor functions with internal and active continuation status is
subsequently assigned to \texttt{efunc.acp\_idx} and collectively refer to
those monitor functions whose values will vary during the subsequent
continuation. Finally, the first $j$ elements of \texttt{efunc.cont\_par\_idx%
}, the indices of monitor functions with internal continuation status and
the remaining elements of \texttt{efunc.cont\_par\_idx} is subsequently
assigned to \texttt{efunc.op\_idx} and collectively refer to those monitor
functions whose values will be included in the output during the subsequent
continuation.

\item The values of $\tilde{\eta}^{\ast }$ are by default computed from the
initial values of $\xi $ and $\rho $ as per $\tilde{\eta}^{\ast }=\tilde{g}%
\left( \xi ,\rho \right) $. These values may be overwritten using the
function \texttt{coco\_set\_parival}. In particular, each call to \texttt{%
coco\_set\_parival} modifies the cell array contained in \texttt{%
efunc.parivals} to include string labels representing the corresponding
monitor functions and the desired numerical values.
\end{itemize}

\subsubsection{Events}

\begin{itemize}
\item Information regarding events associated with monitor functions is
provided to \textsc{coco} using the functions \texttt{coco\_add\_event} and 
\texttt{coco\_set\_dim}. In particular, each call to \texttt{coco\_add\_event%
} appends to \texttt{efunc.events} an event structure that includes a cell
array containing a string identifying the event (in the \texttt{name}
field), a cell array containing an event function handle (in the \texttt{han}
field), a cell array containing string labels for the corresponding monitor
functions (in the \texttt{par} field), an array of numerical trigger values
(in the \texttt{vals} field), an event type (in the \texttt{evlist} field),
and an event signature (in the \texttt{sign} field).

\item An event is labeled a \emph{boundary event} and the string \texttt{%
'BP\_idx'} is assigned to the \texttt{evlist} field if the corresponding
point on the solution manifold lies on the boundary of the computational
domain; is labeled a \emph{terminal event} and the string \texttt{'MX\_idx'}
is assigned to the \texttt{evlist} field if the continuation process should
terminate when the event is triggered; and is otherwise labeled a \emph{%
special event} and the string \texttt{'SP\_idx'} is assigned to the \texttt{%
evlist} field. In the case of terminal events, the \texttt{vals} field
contains exactly one numerical trigger value.

\item \emph{Simple} event signatures trigger when a single scalar-valued
monitor function crosses a predefined numerical value and can be associated
with all three types of events. In the default case, the character \texttt{%
'='} is assigned to the \texttt{sign} field. For boundary events with a
single numerical trigger value and for terminal events, the characters 
\texttt{'\TEXTsymbol{<}'} or \texttt{'\TEXTsymbol{>}'} may alternatively be
assigned to the \texttt{sign} field. The latter choices force the \textsc{%
coco} continuer to check that the initial point lies within the
computational domain.

\item \emph{Grouped} event signatures trigger when at least one of several
scalar-valued monitor functions crosses a predefined numerical value and can
be associated with boundary or special events. In this case, a string of 
\texttt{'='}'s equal in length to the number of elements in the \texttt{par}
and \texttt{vals} fields, respectively, is assigned to the \texttt{sign}
field.

\item Each call to \texttt{coco\_set\_dim} reallocates the information
contained within \texttt{efunc.events} to the structure \texttt{efunc.ev} so
as to include only events in those monitor functions whose indices are
contained in \texttt{efunc.acp\_idx}, \texttt{efunc.regular\_pars}, or 
\texttt{efunc.special\_pars}. In particular, the \texttt{efunc.ev} structure
represents a repacking of the relevant content of \texttt{efunc.events} 
\textbf{across} events rather than within events with the inclusion of
linking information for grouped events signatures.

\item Following execution of \texttt{coco\_set\_dim}, the \texttt{pidx}
field contains an array of indices of the corresponding monitor functions;
the \texttt{vals} field contains an array of the corresponding numerical
trigger values; the \texttt{par\_type} field contains a cell array of
strings identifying the algorithm used to locate the corresponding events (%
\texttt{'continuation'} in the case of events in $\eta $, \texttt{'regular'}
in the case of events in regular monitor functions, and \texttt{'singular'}
in the case of events in singular monitor functions); the \texttt{point\_type%
} field contains a cell array of strings identifying the corresponding
events; the \texttt{evhan} field contains a cell array of event function
handles; the \texttt{evsign} field contains a string of \texttt{'='}'s, 
\texttt{'\TEXTsymbol{<}'}'s, and \texttt{'\TEXTsymbol{>}'}'s associated with
the corresponding events; the \texttt{BP\_idx} field contains an array of
indices of monitor functions involved in boundary events; the \texttt{MX\_idx%
} field contains an array of indices of monitor functions involved in
terminal events; the \texttt{SP\_idx} field contains an array of indices of
monitor functions involved in special events; and the \texttt{evgroup} field
contains a cell array of integer arrays linking elements of the \texttt{pidx}%
, \texttt{vals}, \texttt{par\_type}, \texttt{point\_type}, \texttt{evhan},
and \texttt{evsign} fields associated with grouped event signatures.
\end{itemize}

\subsection{\textsc{coco} event handlers}

For each grouped event signature, the \textsc{coco} continuer requires a
user-provided external event handler that determines the action to be taken
when the corresponding event is triggered. Using a reverse communication
protocol, the event handler is called multiple times by the \textsc{coco}
continuer until it indicates that all possible events in this group have
been processed.

\begin{itemize}
\item A \textsc{coco} event handler must accept three arguments, namely the
global \texttt{opts} structure, a command string, and a data structure. A 
\textsc{coco} event handler must return two arguments, namely a message
structure whose \texttt{action} field determines the subsequent action of
the \textsc{coco} continuer and a modified version of the original data
structure that includes the content of the message structure in the \texttt{%
msg} field. Upon subsequent calls to the event handler, the most recent
content of the message structure is thus available to determine the
subsequent action of the \textsc{coco} continuer.

\item The command string argument corresponds to the state of processing of
the grouped event signature and equals \texttt{'init'} or \texttt{'check'}.
Specifically, when the command string argument equals \texttt{'init'}, the
allowable values for the outgoing message structure's \texttt{action} field
are \texttt{'locate'} and \texttt{'finish'}. Similarly, when the command
string argument equals \texttt{'check'}, the allowable values for the
message structure's \texttt{action} field are \texttt{'add'} and \texttt{%
'reject'}.

\item In the cases that the message structure's \texttt{action} field equals 
\texttt{'locate'}, the message structure's \texttt{idx} field contains an
integer array consisting of the indices of the monitor functions involved in
the corresponding event and the message structure's \texttt{point\_type}
field contains a string identifying the corresponding event.

\item In a typical implementation, a call to the event handler with the
command string \texttt{'init'} and an empty \texttt{msg} field in the data
structure results in the creation of an additional field of the data
structure containing an array of possible events, ordered by priority, that
may have triggered the \textsc{coco} event detection.
\end{itemize}

\subsection{The \textsc{coco} corrector}

The \textsc{coco} corrector implements a single complete step of a line
search method for finding a root of some function $F:\mathbb{R}%
^{n}\rightarrow \mathbb{R}^{n}$. In a default implementation, the \textsc{%
coco} corrector implements a single step of the damped Newton method.
Properties of the corresponding class \texttt{nwtn} control the behavior of
the corrector and store intermediate computational results.

\begin{itemize}
\item The \texttt{nwtn.d} property stores the current nominal correction $%
d\in \mathbb{R}^{n}$ to the current approximation $x\in \mathbb{R}^{n}$
(stored in \texttt{nwtn.x}) of the unique root of $F$ in some neighborhood
of the initial solution guess $x_{0}\in \mathbb{R}^{n}$ (stored in \texttt{%
nwtn.x0}).

\item The \texttt{nwtn.func} property stores a reference to the \texttt{%
opts.(nwtn.func)} class of the global \texttt{opts} structure. The \texttt{%
opts.(nwtn.func).F}, \texttt{opts.(nwtn.func).DFDX}, and \texttt{%
opts.(nwtn.func).linsolve} properties store handles to external methods used
to complete a step of the corrector.

\item The method associated with \texttt{opts.(nwtn.func).F} must accept $x$
as argument and return the \emph{residual} $F\left( x\right) $ (stored in 
\texttt{opts.nwtn.f}).

\item The method associated with \texttt{opts.(nwtn.func).DFDX} must accept $%
x$ as argument and return a suitably defined reference (stored in \texttt{%
opts.nwtn.J}). In a default implementation, \texttt{opts.(nwtn.func).DFDX}
returns the linearization $\partial F\left( x\right) $.

\item The content of \texttt{opts.nwtn.J} is updated provided that the
number of completed corrector steps $i$ (stored in \texttt{nwtn.It}) does
not exceed a maximum value (stored in \texttt{nwtn.ItNW}) if such a value is
set.

\item The method associated with \texttt{opts.(nwtn.func).linsolve} must
accept the outputs of \texttt{opts.(nwtn.func).F} and \texttt{%
opts.(nwtn.func).DFDX} as first and second argument, respectively, and
return $d$. In a default implementation, \texttt{opts.(nwtn.func).linsolve}
employs \textsc{Matlab}'s \texttt{mldivide} operator to solve the linear
equation $\partial F\left( x\right) \cdot d=F\left( x\right) $.

\item Each complete corrector step seeks to yield an improved solution
approximation $x-\gamma d$, where $\gamma $ is a \emph{damping factor}
(stored in \texttt{nwtn.ga}) bounded above by the minimum of the initial
damping factor (stored in \texttt{nwtn.ga0}) and an $x$- and $d$-dependent
maximum given by the ratio $\mathtt{nwtn.MaxStep}\ast \left( 1+\left\|
x\right\| \right) /\left\| d\right\| $, where $\left\| \cdot \right\| $
denotes the standard Euclidean norm.

\item In particular, $\gamma $ is reduced iteratively by a scaling factor
(stored in \texttt{nwtn.al}) as long as the number of iterations $j$ (stored
in \texttt{nwtn.SubIt}) does not exceed a maximum value (stored in \texttt{%
nwtn.SubItMX}) and as long as $\left\| F\left( x-\gamma d\right) \right\| $
is greater than $\left\| F\left( x\right) \right\| $ (stored in \texttt{%
nwtn.norm\_f\_old}).

\item The current solution guess $x$ is characterized as an acceptable
converged solution provided that $\left\| d\right\| $ falls below a preset
tolerance (stored in \texttt{nwtn.TOL}), in which case the value $1$ is
stored in the \texttt{nwtn.accept} property (which is initialized to $0$).

\item Convergence fails if the number of completed corrector steps exceeds a
preset maximum value (stored in \texttt{nwtn.ItMX}). In this case, the 
\textsc{coco} corrector throws an error with identifier \texttt{%
NWTN:NoConvergence}.

\item Provided that the value of the \texttt{nwtn.LogLevel} property exceeds 
$0$, default screen logging of intermediate data includes the current values
of $i$, $j$, $\gamma $, $\left\| d\right\| $, $\left\| F\left( x\right)
\right\| $, $\left\| x\right\| $, and computation times associated with the
residual (stored in \texttt{nwtn.ftm}), the linearization (stored in \texttt{%
nwtn.dftm}) and the correction (stored in \texttt{nwtn.stm}).

\item The \texttt{nwtn.begin\_callback}, \texttt{nwtn.begin\_step\_callback}%
, \texttt{nwtn.end\_step\_callback}, and \texttt{nwtn.end\_callback}
properties store handles to methods that enable extraction and modification
of existing or user-defined properties of the \texttt{opts} class at the
outset of the first call to the corrector, at the outset of each call to the
corrector, at the conclusion of each call to the corrector, and subsequent
to the call to the corrector that results in a converged solution. In a
default implementation, these methods are given by the empty action.

\item The \texttt{nwtn.print\_headline} and \texttt{nwtn.print\_data}
properties store handles to methods that enable screen logging of additional
user-defined output data. In a default implementation, these methods are
given by the empty action.
\end{itemize}

\noindent Tables \ref{table: Newton1}-\ref{table: Newton3} summarize the
properties of the \texttt{nwtn} class and include default values, whenever
appropriate.

\begin{table}[h]
\caption{Properties that influence the corrector behavior.}
\label{table: Newton1}\centering%
\begin{tabular}{|lll|}
\hline
Property & Default value & Explanation \\ \hline
\texttt{ItMX} & $10$ & : max. number of Newton iterations \\ 
\texttt{ItNW} & \texttt{ItMX} & : max. number of linearization updates \\ 
\texttt{SubItMX} & $8$ & : max. number of damping iterations \\ 
\texttt{TOL} & $1.0e-8$ & : convergence criterion \\ 
\texttt{LogLevel} & $1$ & : level of diagnostic output \\ 
\texttt{MaxStep} & $0.1$ & : max. size of Newton step \\ 
\texttt{ga0} & $1.0$ & : initial damping factor \\ 
\texttt{al} & $0.5$ & : scale factor \\ \hline
\end{tabular}%
\end{table}

\begin{table}[h]
\caption{Properties that define function callback handlers.}
\label{table: Newton2}\centering%
\begin{tabular}{|ll|}
\hline
Property & Default value \\ \hline
\texttt{print\_headline} & \texttt{@coco\_default\_print} \\ 
\texttt{print\_data} & \texttt{@coco\_default\_print} \\ 
\texttt{begin\_callback} & \texttt{@coco\_default\_callback} \\ 
\texttt{begin\_step\_callback} & \texttt{@coco\_default\_callback} \\ 
\texttt{end\_callback} & \texttt{@coco\_default\_callback} \\ 
\texttt{end\_step\_callback} & \texttt{@coco\_default\_callback} \\ \hline
\end{tabular}%
\end{table}

\begin{table}[h]
\caption{Properties that store internal data and their default
interpretation.}
\label{table: Newton3}\centering%
\begin{tabular}{|ll|}
\hline
Property & Explanation \\ \hline
\texttt{It} & : number of completed Newton iterations \\ 
\texttt{accept} & : flag if iteration converged \\ 
\texttt{ftm} & : total time spent evaluating residuals \\ 
\texttt{dftm} & : total time spent on evaluating linearizations \\ 
\texttt{stm} & : total time spent on evaluating corrections \\ 
\texttt{SubIt} & : number of completed damping steps \\ 
\texttt{x0} & : initial solution guess \\ 
\texttt{x} & : current solution \\ 
\texttt{f} & : current residual \\ 
\texttt{norm\_f\_old} & : norm of previously computed residual \\ 
\texttt{J} & : current linearization \\ 
\texttt{d} & : current correction \\ 
\texttt{ga} & : current damping factor \\ \hline
\end{tabular}%
\end{table}

Great flexibility is afforded the user to introduce non-default
functionality to the \textsc{coco} corrector. Consider, for example, the
following implementation of Broyden's method.%
\[
B_{i}\cdot d_{i}=F\left( x_{i}\right) \Rightarrow x_{i+1}=x_{i}-\gamma
d_{i}\Rightarrow B_{i+1}=B_{i}-\frac{\left( F\left( x_{i+1}\right) -F\left(
x_{i}\right) +\gamma B_{i}\cdot d_{i}\right) \cdot d_{i}^{T}}{\gamma \left\|
d_{i}\right\| ^{2}} 
\]%
with initial value $B_{0}$ equal to the identity matrix (Ref: NUMERICAL\
RECIPES). To this end, consider the following modifications.

\begin{itemize}
\item Add the \texttt{opts.Broyden.f\_old} and \texttt{opts.Broyden.B}
properties.

\item Let the method associated with \texttt{nwtn.begin\_callback} assign
the identity matrix to \texttt{opts.Broyden.B}.

\item Let the method associated with \texttt{nwtn.begin\_step\_callback}
assign the value of \texttt{nwtn.f} to \texttt{opts.Broyden.f\_old}.

\item Let the method associated with \texttt{opts.(nwtn.func).DFDX} return
the current value of \texttt{opts.Broyden.B}.

\item Let the method associated with \texttt{nwtn.end\_step\_callback}
update the value of \texttt{opts.Broyden.B}.
\end{itemize}

\subsection{Manifold continuation}

\textsc{coco} is compatible with a general-purpose manifold-covering
algorithm but provides an explicit implementation only for the special case
of a one-dimensional solution manifold. A manifold cover is here taken to
consist of a consistent network $\mathcal{N}$ of linearized charts $\left(
u,t\right) $, each of which is composed of a point $u=\left( 
\begin{array}{ccc}
\xi & \rho & \eta%
\end{array}%
\right) $ on the manifold and a basis of the corresponding tangent space $t$%
. Specifically, the collection of nodes of the network constitutes a
manifold atlas, whereas the network connectivity distinguishes between
internal, external, and boundary charts and ensures that further
continuation is not performed into already computed regions of the atlas.
Consistency is imposed through conditions on the network connectivity that
respect a natural manifold meshing with reasonable numerical bounds on
curvature.

Given an initial manifold cover, \textsc{coco} proceeds to select an
external chart $\left( u,t\right) $ and to formulate a \emph{predictor} $v$,
such that $v-u$ lies in the tangent space $t$. In particular, the predictor
is chosen so as to ensure that there exists a consistent network $\mathcal{N}%
^{\prime }\supset \mathcal{N}$ that contains $\left( v,t\right) $. Using $v$
as initial guess, the application of an iterative root-solver to the \emph{%
corrector} (cf. \ref{eq: Continuation}) then yields a new point $u^{\prime }$
on the solution manifold.

\textsc{coco} accepts this point provided that there exists a consistent
network $\mathcal{N}^{\prime }\supset \mathcal{N}$ that contains the
corresponding chart $\left( u^{\prime },t^{\prime }\right) $ and that this
point is within the computational domain. In this case, the network $%
\mathcal{N}$ is replaced by $\mathcal{N}^{\prime }$. On the other hand, if
either condition fails, the point $u^{\prime }$ is rejected and the step is
repeated with a refined predictor. In particular, in the case that $%
u^{\prime }$ lies outside the computational domain, an attempt is made to
locate a co-dimension-one boundary crossing by replacing one of the
components of $h$ in (\ref{eq: Continuation}) with a suitable boundary
condition.

After each successful predictor-corrector step, \textsc{coco} checks the
components of $\eta $ for co-dimension-one crossings of predefined numerical
values along newly formed one-dimensional edges of the covering network.
Such \emph{special points} are subsequently located on the solution manifold
by replacing one of the components of $h$ in (\ref{eq: Continuation}) with
the condition $\eta _{i}=\eta _{i}^{\ast }$ for some $i$.

In addition, \textsc{coco} also checks the components of an additional set
of not-twice differentiable monitoring functions for co-dimension-one
crossings of predefined numerical values along newly formed one-dimensional
edges of the covering network. In this instance, special points are
subsequently located with a solver that combines bisection and root-solver
iterations to an extent determined by the degree of regularity of the
special point.

A special point is regular if there exists a rank-$j$ matrix $V$, such that $%
V^{T}\cdot \left( u^{\prime }-u\right) \neq 0$ and such that the
linearization at the special point of the corrector (\ref{eq: Continuation})
with $h$ as in (\ref{eq: hlinear}) is non-singular. In this case, let $%
\lambda =kV^{T}\cdot \left( u^{\prime }-u\right) $ for $k\in \left(
0,1\right) $ and perform bisection on $k$ combined with the application of
an iterative root solver to the corrector (\ref{eq: Continuation}) to locate
the special point. Co-dimension-one crossings of predefined values of the
maximum of a set of functions of $\xi $ and $\rho $ are examples of regular
points.

In the singular case, let $u_{k}=\left( 1-k\right) u+ku^{\prime }$ with $%
k\in \left( 0,1\right) $ and perform bisection on $k$ to locate a point on
the straight-line segment between $u$ and $u^{\prime }$ at which the
corresponding monitor function vanishes. This point may serve as an initial
guess for a further extended function $F$ for which the special point is
regular. Classical examples of the singular case are pitchfork and
transcritical bifurcation points. In particular, \textsc{coco} treats events
associated with simultaneous crossings of predefined numerical values of
multiple scalar-valued monitor functions as singular.

\subsection{The \textsc{coco} continuer}

The \textsc{coco} continuer implements a one-parameter covering algorithm
for the solution manifold associated with the equation $\left. \tilde{F}%
\right| _{\mathcal{M}}\left( u\right) =0$, where $\left. \tilde{F}\right| _{%
\mathcal{M}}:\mathbb{R}^{n+m+k+1}\rightarrow \mathbb{R}^{n+m+k}$. Properties
of the class \texttt{cont} control the behavior of the covering algorithm
and store intermediate computational results. Properties of the class 
\texttt{state} control the behavior of the associated finite-state machine.

\subsubsection{Manifold atlas}

\begin{itemize}
\item The \texttt{cont.charts} property contains a stack of active charts.

\item The \texttt{x} property of each chart contains the corresponding value
of $u=\left( \xi ,\rho ,\eta \right) $. The \texttt{t} property contains a
corresponding tangent vector to the solution manifold. The \texttt{p}
property contains the corresponding values of all defined monitoring
functions. The \texttt{init} property is nonzero in the case of an initial,
unprocessed, chart and zero otherwise. If not empty, the \texttt{pt\_type}
contains the label \texttt{'EP'} in the case of a chart on the boundary of
the domain of continuation, the label \texttt{'MX'} in the case of a chart
at which continuation must terminate (e.g., beyond which continuation was
not possible), or a user-defined label in the case of a chart associated
with a user-defined event. The \texttt{ep\_flag} property is empty by
default; equals $1$ in the case of a boundary chart; and equals $2$ in the
case of a terminal chart.
\end{itemize}

\subsubsection{Finite-state machine}

\begin{itemize}
\item The \textsc{coco} continuer exits the finite-state machine when the
value $1$ is stored in the \texttt{cont.accept} property (which is
initialized to $0$).

\item The \texttt{cont.state} property contains the current state of the
finite-state machine and equals \texttt{init}, \texttt{init\_chart\_list}, 
\texttt{predict}, \texttt{correct}, \texttt{check\_solution}, \texttt{%
refine\_step}, \texttt{locate\_boundary\_events}, \texttt{%
check\_terminate\_events}, \texttt{locate\_special\_events}, \texttt{%
locate\_events}, \texttt{locate\_multi}, \texttt{locate\_cont}, \texttt{%
locate\_reg}, \texttt{locate\_sing}, \texttt{locate\_BP\_warning}, \texttt{%
locate\_MX\_warning}, \texttt{locate\_SP\_warning}, \texttt{add\_BP}, 
\texttt{add\_MX}, \texttt{add\_SP}, \texttt{insert\_points}, or \texttt{%
update}. Each value \texttt{\%\%\%} of the \texttt{cont.state} property is
associated with three properties \texttt{state.\%\%\%}, \texttt{%
state.bcb\_\%\%\%}, and \texttt{state.ecb\_\%\%\%} of the \texttt{state}
class. The first of these contains a handle to a predefined method
associated with the corresponding state. The latter two contain handles to a
pair of user-defined callback functions that enable extraction and
modification of existing or user-defined properties of the \texttt{opts}
class before and after, respectively, each call to the corresponding
predefined method. The diagram in FIG\ XXX represents the flow through the
finite-state machine.

\item The \texttt{cont.next\_state} property contains the state of the
finite-state machine subsequent to a call to the method associated with the 
\texttt{correct} state at which convergence is achieved. The \texttt{%
cont.err\_state} property contains the state of the finite-state machine
subsequent to a call to the method associated with the \texttt{correct}
state at which the \texttt{NWTN:NoConvergence} error is thrown. In
particular, if \texttt{cont.err\_state} is empty, execution terminates with
the \texttt{NWTN:NoConvergence} error.

\item The \texttt{cont.locate\_next\_state}, \texttt{cont.locate\_warn\_state%
}, and \texttt{cont.locate\_add\_state} properties contains the XXXX state
of the finite-state machine subsequent to a call to the method associated
with the \texttt{correct} state at which convergence is achieved. XXXX
\end{itemize}

\subsubsection{Initialization}

\begin{itemize}
\item The method associated with the \texttt{init} state assigns the initial
guess $u_{0}$ (contained in \texttt{cont.u0}) to \texttt{nwtn.x0} and sets
the \texttt{nwtn.func} property to refer to the \texttt{xfunc} class. The 
\texttt{xfunc.fixpar\_idx}, \texttt{xfunc.fixpar\_val}, and \texttt{%
xfunc.mode} properties are set to the index corresponding to the primary
continuation parameter $\hat{\eta}$, the numerical value $\hat{\eta}^{\ast }$%
, and $1$, respectively, so as to reflect the choice $h\left( u\right) =\hat{%
\eta}-\hat{\eta}^{\ast }$ in (\ref{eq: Continuation}). The \texttt{cont.state%
}, \texttt{cont.next\_state}, and \texttt{cont.err\_state} properties are
subsequently set to \texttt{correct}, \texttt{init\_chart\_list}, and the
empty array, respectively.

\item The method associated with the \texttt{correct} state calls the 
\textsc{coco} corrector repeatedly until the \texttt{NWTN:NoConvergence}
error is thrown or until convergence is achieved.

\item The method associated with the \texttt{init\_chart\_list} state
initiates the chart stack and subsequently sets the \texttt{cont.state}
property to \texttt{predict}. In particular, the chart stack contains one
chart with \texttt{pt\_type} equal to \texttt{EP} and associated with the
converged point $u$ (assigned to \texttt{cont.u0}) and a tangent direction
corresponding to initially increasing values of $\hat{\eta}$ provided that
the correponding maximum number of continuation steps (stored in the first
element of the \texttt{cont.ItMX} array) is positive; one chart with \texttt{%
pt\_type} equal to \texttt{EP} and associated with the converged point $u$
and a tangent direction corresponding to initially decreasing values of $%
\hat{\eta}$ provided that the correponding maximum number of continuation
steps (stored in the second element of the \texttt{cont.ItMX} array) is
positive; both charts provided that both entries of the \texttt{cont.ItMX}
array are positive; or is an empty cell array if neither entry of the 
\texttt{cont.ItMX} array is positive;

\item If at least one of the entries of the \texttt{cont.ItMX} array is
positive, the method associated with the \texttt{init\_chart\_list} state
throws an error if two or more inequalities associated with relevant
boundary events are violated or if one such inequality is violated and the
value of the corresponding monitor function deviates from the trigger value
by more than a given tolerance ($10$ times the value stored in \texttt{%
opts.nwtn.TOL}). Here, an inequality associated with a boundary event is
satisfied provided that the corresponding monitor function is greater than
the trigger value in the case that the \texttt{sign} field is \texttt{%
\TEXTsymbol{<}} and less than the trigger value in the case that the \texttt{%
sign} field is \texttt{\TEXTsymbol{>}}. Boundary events are relevant
provided that the corresponding event handler, when defined, does not return
the \texttt{reject} message when called with the \texttt{check} flag and
data corresponding to the converged point $u$.

\item If, instead, at least one of the entries of the \texttt{cont.ItMX}
array is positive, one inequality associated with a relevant boundary event
is violated, the value of the corresponding monitor function deviates from
the trigger value by less than the given tolerance, the corresponding 
\texttt{par\_type} is \texttt{continuation} or \texttt{regular}, and a
finite-difference estimate of the derivative of the monitor function along
the tangent direction deviates from zero by more than the given tolerance,
the method associated with the \texttt{init\_chart\_list} state modifies the
user-defined content of the \texttt{cont.ItMX} array prior to chart-stack
initiation. In particular, if the corresponding \texttt{sign} field is 
\texttt{\TEXTsymbol{>}} (\texttt{\TEXTsymbol{<}}), the first (second) entry
of the \texttt{cont.ItMX} array is set to zero. In this case, the \texttt{%
chart.ignore\_at} field is set equal to the index of the corresponding
monitor function.
\end{itemize}

\subsubsection{Prediction and correction}

\begin{itemize}
\item Given an empty chart stack, the method associated with the \texttt{%
predict} state sets \texttt{cont.accept} to $1$ and terminates.

\item Given a non-empty chart stack, the method associated with the \texttt{%
predict} state sets the \texttt{xfunc.u0}, \texttt{xfunc.t}, and \texttt{%
xfunc.mode} properties to $u^{\ast }=$\texttt{chart.x}, $t=$\texttt{chart.t}%
, and $1$, respectively, so as to reflect the choice $h\left( u\right)
=t^{T}\cdot \left( u-u^{\ast }\right) +\lambda \left\| t\right\| ^{2}$ in (%
\ref{eq: Continuation}) and constructs a predictor $u$ (assigned to \texttt{%
nwtn.x0}) separated from the point $u^{\ast }$ by a multiple $\lambda $
(stored in \texttt{xfunc.h}) of $t$. The value of $\lambda $ is reduced
iteratively from its current value (initialized at \texttt{cont.h0} for an
initial, unprocessed, chart) by a scaling factor $\phi _{\min }$ (stored in 
\texttt{cont.h\_fac\_min}) as long as the norm of the residual (evaluated by
the method associated with the \texttt{opts.xfunc.F} property) exceeds an
upper bound (stored in \texttt{cont.MaxRes}) and $\lambda $ exceeds a
minimum value $\lambda _{\min }$ (stored in \texttt{cont.h\_min}) and is set
to equal this minimum value if scaling results in a number smaller than the
minimum. For an initial, unprocessed chart, \texttt{cont.ptlist} is
initialized to a cell array containing the chart and \texttt{cont.current\_pt%
} it set to $1$. The \texttt{cont.state}, \texttt{cont.next\_state}, and 
\texttt{cont.err\_state} properties are subsequently set to \texttt{correct}
(see above), \texttt{check\_solution}, and \texttt{refine\_step},
respectively.

\item The method associated with the \texttt{refine\_step} state assigns $%
\max \left( \lambda _{\min },\phi _{\min }\lambda \right) $ to $\lambda $
and sets the \texttt{cont.state} property to \texttt{predict} unless $%
\lambda \leq \lambda _{\min }$, in which case the \texttt{cont.state}
property is set equal to \texttt{add\_MX}.

\item If the angle between the tangent vector associated with a converged
point $u$ and the tangent vector associated with the current chart exceeds
an upper bound (equal to the product of the values stored in \texttt{%
cont.h\_fac\_max} and \texttt{cont.arc\_alpha}) and $\lambda >\lambda _{\min
}$, the method associated with the \texttt{check\_solution} state assigns $%
\max \left( \lambda _{\min },\phi _{\min }\lambda \right) $ to $\lambda $
and sets the \texttt{cont.state} property to \texttt{predict}.

\item If either condition is violated, the method associated with the 
\texttt{check\_solution} state appends the current chart, with \texttt{%
pt\_type} equal to \texttt{EP} and \texttt{ep\_flag} equal to $1$ in the
case that the number of completed iterations (stored in \texttt{cont.It})
exceeds the corresponding entry of \texttt{cont.ItMX} and with empty \texttt{%
pt\_type} and \texttt{ep\_flag} equal to $0$ otherwise, to the cell array
contained in \texttt{cont.ptlist} and sets \texttt{cont.current\_pt} to $2$.
Moreover, the \texttt{cont.events} structure is initialized so that the 
\texttt{u0}, \texttt{u1}, \texttt{us0}, and \texttt{us1} fields contain the
points and tangent vectors, respectively, for the two charts contained in 
\texttt{cont.ptlist}; the \texttt{us} field contains a unit vector parallel
to and directed along the average of the two tangent vectors; and the 
\texttt{dh} field contains the projected distance between the two points
along the direction spanned by \texttt{us}. The content of the \texttt{u0}
and \texttt{us} fields is also stored in the \texttt{xfunc.u0} and \texttt{%
xfunc.t}\textbf{\ }properties. The \texttt{cont.state} property is
subsequently set to \texttt{locate\_boundary\_events}.
\end{itemize}

\subsubsection{Event handling}

\begin{itemize}
\item The method associated with the \texttt{locate\_boundary\_events} state
appends the \texttt{cont.events} structure; sets the \texttt{%
locate\_next\_state}, \texttt{locate\_warn\_state}, and \texttt{%
locate\_add\_state} properties to \texttt{check\_terminate\_events}, \texttt{%
locate\_BP\_warning}, and \texttt{add\_BP}, respectively; and sets the 
\texttt{cont.state} property to \texttt{check\_terminate\_events} if \texttt{%
cont.events.evlist} is empty and to \texttt{locate\_events}, otherwise. In
particular, the \texttt{cont.events.ev1} and \texttt{cont.events.ev2} fields
are assigned arrays $e_{1}$ and $e_{2}$ consisting of the values of all
defined monitor functions at the \texttt{u0} and \texttt{u1} points.
Moreover, the \texttt{cont.events.evlist} field contains those indices
contained in the \texttt{efunc.ev.BP\_idx} array but not in the \texttt{%
ignore\_at} field of the first or second charts in \texttt{cont.ptlist}; for
which $e_{1i}\neq 0$ and $\left| e_{i1}-e_{i2}\right| $ exceeds a given
tolerance ($10$ times the value stored in \texttt{opts.nwtn.TOL}); and for
which $e_{1i}e_{2i}\leq 0$ in the case of initially increasing values of $%
\hat{\eta}$ and $e_{1i}e_{2i}<0$ in the case of initially decreasing values
of $\hat{\eta}$.

\item The method associated with the \texttt{check\_terminate\_events} state
truncates the \texttt{cont.ptlist} array to its first two elements, appends
the \texttt{cont.events} structure; and sets the \texttt{cont.state}
property to \texttt{locate\_special\_events} if \texttt{cont.events.evlist}
is empty and to \texttt{add\_MX}, otherwise. In particular, the \texttt{%
cont.events.ev1} and \texttt{cont.events.ev2} fields are assigned arrays $%
e_{1}$ and $e_{2}$ of the values of all defined monitor functions at the
points given by the content of the \texttt{u0} and \texttt{u1} fields, which
in turn are assigned to equal the \texttt{x} fields of the first and second
charts in \texttt{cont.ptlist}, respectively. Moreover, the \texttt{%
cont.events.evlist} field contains those indices contained in the \texttt{%
efunc.ev.MX\_idx} array, for which $e_{1i}\neq 0$ and $\left|
e_{i1}-e_{i2}\right| $ exceeds a given tolerance ($10$ times the value
stored in \texttt{opts.nwtn.TOL}); and for which $e_{1i}e_{2i}\leq 0$ in the
case of initially increasing values of $\hat{\eta}$ and $e_{1i}e_{2i}<0$ in
the case of initially decreasing values of $\hat{\eta}$.

\item The method associated with the \texttt{locate\_special\_events} state
appends the \texttt{cont.events} structure; sets the \texttt{%
locate\_next\_state}, \texttt{locate\_warn\_state}, and \texttt{%
locate\_add\_state} properties to \texttt{insert\_points}, \texttt{%
locate\_SP\_warning}, and \texttt{add\_SP}, respectively; and sets the 
\texttt{cont.state} property to \texttt{insert\_points} if \texttt{%
cont.events.evlist} is empty and to \texttt{locate\_events}, otherwise. In
particular, the \texttt{cont.events.ev1} and \texttt{cont.events.ev2} fields
are assigned arrays $e_{1}$ and $e_{2}$ of the values of all defined monitor
functions at the points given by the \texttt{u0} and \texttt{u1} points.
Moreover, the \texttt{cont.events.evlist} field contains those indices
contained in the \texttt{efunc.ev.SP\_idx} array, for which $e_{1i}\neq 0$
and $\left| e_{i1}-e_{i2}\right| $ exceeds a given tolerance ($10$ times the
value stored in \texttt{opts.nwtn.TOL}); and for which $e_{1i}e_{2i}\leq 0$
in the case of initially increasing values of $\hat{\eta}$ and $%
e_{1i}e_{2i}<0$ in the case of initially decreasing values of $\hat{\eta}$.

\item If the \texttt{cont.events} structure is empty, the method associated
with the \texttt{add\_MX} state stores the last obtained chart and a copy
with \texttt{pt\_type} equal to \texttt{MX} and \texttt{ep\_flag} equal to $%
2 $ in the \texttt{cont.ptlist} and sets the \texttt{cont.state} property to 
\texttt{insert\_points}.

\item After logging of the content of cont.ptlist, the method associated
with the \texttt{insert\_points} state deletes all but the last chart
contained in cont.ptlist, sets cont.current\_pt to $1$, and sets the \texttt{%
cont.state} property to \texttt{update}.
\end{itemize}

\subsubsection{Updating}

\begin{itemize}
\item The method associated with the \texttt{update} state

\item The \texttt{cont.efunc} and \texttt{cont.xfunc }properties store
references to the \texttt{opts.(cont.efunc)} and \texttt{opts.(cont.xfunc)}
classes of the global \texttt{opts} structure.

\item The \texttt{opts.(cont.efunc).F}, \texttt{opts.(cont.efunc).DFDX}, and 
\texttt{opts.(cont.efunc).linsolve} properties store function handles for
computing the \emph{residual} $F\left( x,p\right) $ (stored in), a
corresponding linearization $\left( 
\begin{array}{cc}
\partial _{x}F & \partial _{p}F%
\end{array}%
\right) $ (stored in), and for solving the linear equation $\partial
_{x}F\cdot t_{x}+\tau \partial _{p_{1}}F+\partial _{p_{2\ldots n}}F\cdot
t_{p}=0$, respectively, for some function $F:\mathbb{R}^{n}\times \mathbb{R}%
^{m}\rightarrow \mathbb{R}^{n+m-1}$.

\item Here, $x$ and $p$ are the components of a vector stored in \texttt{%
cont.u} containing the current solution guess for a point on the root
manifold in the vicinity of the initial point (stored in \texttt{cont.u0}).
In particular, \texttt{cont.xidx} and \texttt{cont.pidx} store the index
references to the corresponding positions in \texttt{cont.u}.
\end{itemize}

\section{Toolbox development}

\section{Appendix}
\begin{verbatim}
opts = [];
opts = coco_add_func(opts, @plane, 'internal', 'H');
opts = coco_add_event(opts, 'UZ', 'H',  0);
opts = coco_set(opts, 'defaults', 'CleanData', 1);
bd1 = coco(opts, 'alcont', @cone, 'x', 'isol', 'sol', ...
\qquad \qquad \qquad \qquad \lbrack 0.4], [0;0.5;0], 'PAR(2)', [0.1, 2]);
 
z = [bd1{2:end,10}];
p = [bd1{2:end,11}];
x = p(1,:);
y = p(2,:);
plot3(x, y, z, 'b.-');
grid on
view([95 10]);
drawnow
 
idx = find(strcmp('UZ', { bd1{2:end,4} }));
lab = [ bd1{idx+1,6} ];
hold on
plot3(x(idx), y(idx), z(idx), 'go', 'LineWidth', 2, 'MarkerSize', 10);
hold off
drawnow
 
opts = coco_add_event(opts, 'UZ', 'PAR(3)', (-3:3)*pi/32);
bd2 = coco(opts, 'alcont', @cone, 'sec', 'sol', 'sol', ...
\qquad \qquad \qquad \qquad 'x', lab, {'PAR(3)' 'PAR(2)'}, [-pi/8 pi/8]);
 
z = [bd2{2:end,10}];
p = [bd2{2:end,11}];
x = p(1,:);
y = p(2,:);
idx = find(strcmp('UZ', { bd2{2:end,4} }));
lab = [ bd2{idx+1,6} ];
hold on
plot3(x(idx), y(idx), z(idx), 'ro', 'LineWidth', 2);
hold off
drawnow
 
opts = coco_set(opts, 'cont', 'ItMX', [30 30]);
for i=1:numel(lab)
\qquad run = sprintf('sec_%d', lab(i));
\qquad bd3 = coco(opts, 'alcont', @cone, run, 'sol', 'sol', ...
\qquad \qquad \qquad \qquad \qquad 'sec', lab(i), {'PAR(1)' 'PAR(2)'}, [-2 2]);
 
\qquad z = [bd3{2:end,10}];
\qquad p = [bd3{2:end,11}];
\qquad x = p(1,:);
\qquad y = p(2,:);
\qquad hold on
\qquad plot3(x, y, z, 'g.-');
\qquad hold off
\qquad grid on;
\qquad drawnow
end
\end{verbatim}

\subsection{Empty zero problem}
\begin{verbatim}
opts = [];
opts = coco_add_func(opts, @cone, 'active', 'G');
opts = coco_set_parival(opts, 'G', 0);
opts = coco_xchg_pars(opts, 'G', 'PAR(1)');
opts = coco_add_func(opts, @plane, 'active', 'H');
opts = coco_add_event(opts, 'UZ', 'H', 0);
opts = coco_set(opts, 'cont', 'LogLevel', [1 0]);
bd1 = coco(opts, 'alcont', [], 'ray1', 'isol', 'sol', ...
\qquad \qquad \qquad \qquad \lbrack ], [0.4;0;0.5;0], 'PAR(3)', [0.1 2]);
 
p = [bd1{2:end,11}];
z = p(1,:);
x = p(2,:);
y = p(3,:);
plot3(x, y, z, 'b.-');
grid on
drawnow
 
idx = find(strcmp('UZ', { bd1{2:end,4} }));
lab = [ bd1{idx+1,6} ];
hold on
plot3(x(idx), y(idx), z(idx), 'go', 'LineWidth', 2);
hold off
drawnow
 
opts = coco_set_parival(opts, []); % reset initial values for subsequent runs
opts = coco_xchg_pars(opts, 'G', 'H');
opts = coco_add_event(opts, 'UZ', 'G', 0);
bd2 = coco(opts, 'alcont', [], 'sec', 'sol', 'sol', ...
\qquad \qquad \qquad \qquad 'ray1', lab, 'PAR(3)', [-2 2]);
 
p = [bd2{2:end,11}];
z = p(1,:);
x = p(2,:);
y = p(3,:);
hold on
plot3(x, y, z, 'k.-');
hold off
idx = find(strcmp('UZ', { bd2{2:end,4} }));
lab = [ bd2{idx+1,6} ];
hold on
plot3(x(idx), y(idx), z(idx), 'ro', 'LineWidth', 2);
hold off
drawnow
 
opts = coco_xchg_pars(opts, 'G', 'H');
bd3 = coco(opts, 'alcont', [], 'ray2', 'sol', 'sol', ...
\qquad \qquad \qquad \qquad 'sec', lab(1), 'PAR(3)', [-2 -0.1]);
 
p = [bd3{2:end,11}];
z = p(1,:);
x = p(2,:);
y = p(3,:);
hold on
plot3(x, y, z, 'b.-');
hold off
drawnow
\end{verbatim}

\newpage 

\begin{thebibliography}{99}
\bibitem{Allgower & Georg} E.L. Allgower and K. Georg, \emph{Numerical
continuation methods: An introduction}, Springer-Verlag, 1990.

\bibitem{Champneys et al 1996} A.R. Champneys, Y.A. Kuznetsov, and B.
Sandstede, ``A numerical toolbox for homoclinic bifurcation analysis,'' 
\emph{Internat. J. Bifur. Chaos Appl. Sci. Engrg. }\textbf{6(5)}, pp.
867--887, 1996.

\bibitem{Dercole & Kuznetsov 2005} F. Dercole and Y.A. Kuznetsov, ``\textsc{%
slidecont}: An \textsc{auto} \textsc{97} driver for bifurcation analysis of
Filippov systems,'' \emph{ACM Trans. Math. Software} \textbf{31(1)}, pp.
95--119, 2005.

\bibitem{Dhooge et al 2003} A. Dhooge, W. Govaerts, and Y.A. Kuznetsov, ``%
\textsc{matcont}: A \textsc{matlab} package for numerical bifurcation
analysis of ODEs,'' \emph{ACM Trans. Math. Software} \textbf{29(2)}, pp.
141--164, 2003.

\bibitem{Doedel et al 2005} E.J. Doedel, W. Govaerts, Y.A. Kuznetsov, and A.
Dhooge, ``Numerical continuation of branch points of equilibria and periodic
orbits,'' \emph{J. Bifur. Chaos Appl. Sci. Engrg.} \textbf{15(3)}, pp.
841--860, 2005.

\bibitem{Engelborghs et al 2002} K. Engelborghs, T. Luzyanina, and D. Roose,
``Numerical bifurcation analysis of delay differential equations using 
\textsc{dde-biftool},'' \emph{ACM Trans. Math. Software} \textbf{28(1)}, pp.
1--21, 2002.

\bibitem{Govaerts et al 2000} W. Govaerts, Y.A. Kuznetsov, and A. Dhooge,
``Numerical continuation of bifurcations of limit cycles in \textsc{matlab}%
,'' \emph{SIAM J. Sci. Comput.} \textbf{27(1)}, pp. 231--252, 2005.

\bibitem{Govaerts 2000} W. Govaerts, ``Numerical bifurcation analysis for
ODEs,'' \emph{J. Comput. Appl. Math.} \textbf{125(1-2)}, pp. 57--68, 2000.

\bibitem{Guckenheimer 1998} J. Guckenheimer, ``Computer simulation and
beyond--for the 21st century,'' \emph{Notices Amer. Math. Soc.} \textbf{45(9)%
}, pp. 1120--1123, 1998.

\bibitem{Keller 1987} H.B. Keller, \emph{Lectures on Numerical Methods in
Bifurcation Problems}, Springer-Verlag, New York, 1987.

\bibitem{Salinger et al 2005} A.G. Salinger, E.A. Burroughs, R.P. Pawlowski,
E.T. Phipps, and L.A. Romero, ``Bifurcation tracking algorithms and software
for large scale applications,'' \emph{J. Bifur. Chaos Appl. Sci. Engrg.} 
\textbf{15(3)}, pp. 1015--1032, 2005.

\bibitem{Szalai et al 2006} R. Szalai, G. Stepan, and S.J. Hogan,
``Continuation of bifurcations in periodic delay-differential equations
using characteristic matrices,'' \emph{SIAM J. Sci. Comput.} \textbf{28(4)},
pp. 1301--1317, 2006.

\bibitem{Thota and Dankowicz 2008} P. Thota and H. Dankowicz, ``On a
Boundary-Value Formulation for the Continuation of Solution Trajectories in
Hybrid Dynamical Systems and its Implementation in the Software Toolbox
TC-HAT $\widehat{TC}$,'' \emph{SIAM J. App. Dyn. Syst.}, to appear, 2008.

\bibitem{Wulff and Schebesch 2006} C. Wulff and A. Schebesch, ``Numerical
continuation of symmetric periodic orbits,'' \emph{SIAM J. Appl. Dyn. Syst.} 
\textbf{5(3)}, pp. 435--475, 2006.
\end{thebibliography}

\end{document}
